\chapter{Newton's Laws of Motion}

The following is an excerpt from \cite[p.~20]{newton}, written by Sir Isaac Newton himself:

\begin{itemize}
  \item[\textbf{Law I.} ]
      \emph{
        Every body perserveres in its state of rest, or of uniform motion in a right
        line, unless it is compelled to change that state by forces impres'd thereon.
      }

      Projectiles persevere in their motions, so far as they are not retarded by
      the resiliance of air, or impell'd downwards by the force of gravity. A
      top, whole parts by their cohesion are preptually drown aside from
      rectilinear motions, does not cease its rotation, otherwise than it is
      retarded by air. The greater bodies of the Planets and Comets, meeting
      with less resistance in more free spaces, preserve their motions both
      progressive and circular for a much longer time.
      %\hfill\cite[p.~19]{newton}
  \item[\textbf{Law II.} ]
      \emph{
        The alteration of motion is ever proportional to the motive force
        impres'd; and is made in the direction of the right line in which that
        force is impres'd.
      }

      If any force generates a motion, a double force will generate double
      motion, a triple force triple the motion, whether that force be impres'd
      altogether and at once, or gradually and successively. And this motion
      (being always directed the same way with the generating force) if the body
      moved before, is added to or subducted from the former motion, according
      as they directly conspire with or are directly contrary to each other; or
      obliquely joyned, when they are oblique, so as to produce a new motion
      compounded from the determination of both.
      %\hfill\cite[p.~20]{newton}
  \item[\textbf{Law III.} ]
      \emph{
        To every Action there is always opposed an equal Reaction: or the mutual
        actions of two bodies upon each other are always equal, and directed to
        contrary parts.
      }

      Whatever draws or presses another is as much drawn or pressed by that
      other. If you press a stone with your finger, the finger is also pressed by
      the stone. If a horse draws a stone tyed to a rope, the horse (if I may so
      say) will be equally drawn back toward the stone: for the distended rope,
      by the same endeavour to relax or unbend it self, will draw the horse as
      much towards the stone, and will obstruct the progress of the stone as
      much as it advances that of the other. If a body impinge upon another, and
      by its force change the motion of the other; that body also (because of
      the equality of the mutual pressure) will undergo an equal change, in its
      own motion, towards the contrary part. The changes made by these actions
      are qual, not in the velocities, but in the motions of bodies; that is to
      say, if the bodies are not hinder'd by any other impediments. For because
      the motions are equally changed, the changes of the velocities made towars
      contrary parts, are reciprocally proportional to the bodies. This Law
      takes place also in Attractions, as will be proved in the next Scholium.
\end{itemize}
% \section{Kinematic Equations}
%
% \textbf{Average acceleration} is defined as the \emph{change in velocity} over
% divided by the \emph{change in time} during which the change in velocity occurs.
% Average acceleration is not normally accurate enough for our purposes, however,
% so we will speak more frequently about the \textbf{Instantaneous acceleration},
% given by
% \begin{equation}
%   \label{eq:acceleration}
%   \vec a(t) = \frac{\ud \vec v}{\ud t}
% \end{equation}
% as a function of time.
%
% For one direction, $\hat x$, we can use simple calculus to derive our basic
% kinematic equations. We can then generalize this to apply to all of kinematics.
%
% Take equation \eqref{eq:acceleration} and rewrite it for just the $\hat x$
% direction.
% \[ a_x = \frac{\ud v_x}{\ud t} \]
% Rewrite this as $\ud v_x = a_x \ud t$ and take the integral of both sides from
% $0$ to $t$, our final time.
% \[ \int \ud v_x = \int^t_0 a_x \ud t \]
% Assume acceleration is constant, giving us
% \begin{align}
%   \nonumber v_x \bigg|^t_0 &= a \int^t_0 \ud t \\
%   v_{xf} - v_{xi} &= a_x(t-0) = a_x t \\
%   \intertext{We can get an equation for velocity if we add $v_{xi}$ to each
% side:}
%   v_{xf} &= a_x t + v_{xi} \label{eq:finalvelocity}
% \end{align}
%
% Now we take the definition for \textbf{instantaneous
% velocity}\index{instantaneous velocity}
% \[ v_x = \frac{\ud x}{\ud t} \]
% and rearrange it and write it as an integral, just as we did for acceleration:
% \[ x_f - x_i = \int^t_0 v_x \ud t \]
% Now we substitute equation \ref{eq:finalvelocity} into this integral as $v_x$
% to get
% \begin{align}
%   \nonumber
%   x_f - x_i &= \int^t_0 (v_{xi} + a_x t) \ud t \\
%   \intertext{Because an integral of a sum is a sum of integrals, we can say}
%   \nonumber &= \int^t_0 v_{xi} \ud t + a_x \int^t_0 t \ud t \\
%   \nonumber &= v_{xi}(t-0)+a_x \left( \frac{t^2}{2}-0 \right) \\
%   x_f - x_i &= v_{xi} t + \frac{1}{2}a_x t^2\\
%   x_f &= \frac{1}{2}a_x t^2 + v_{xi} t + x_i
% \end{align}
