\part{Physics}
\setcounter{section}{0}
\section*{\emph{Syst\'eme International} Prefixes}
\begin{table}[h]
  \centering
  \begin{tabular}{llr}
    \textbf{Power} & \textbf{Prefix} & \textbf{Abbreviation} \\ \hline
    $10^{-24}$ & yocto & y \\
    $10^{-21}$ & zepto & z \\
    $10^{-18}$ & atto & a \\
    $10^{-15}$ & femto & f \\
    $10^{-12}$ & pico & p \\
    $10^{-9}$ & nano & n \\
    $10^{-6}$ & micro & $\upmu$ \\
    $10^{-3}$ & milli & m \\
    $10^{-2}$ & centi & c \\
    $10^{-1}$ & deci & d \\
    $10^{3}$ & kilo & k \\
    $10^{6}$ & mega & M \\
    $10^{9}$ & giga & G \\
    $10^{12}$ & tera & T \\
    $10^{15}$ & peta & P \\
    $10^{18}$ & exa & E \\
    $10^{21}$ & zeta & Z \\
    $10^{24}$ & yotta & Y \\ \hline
  \end{tabular}
  \caption{Prefixes for powers of ten.}
  \label{tab:si_prefixes}
\end{table}

\section*{Significant Figures}

When computing results from measured numbers, pay careful attention to the
number of \textbf{significant figures}\index{significant figures}. When
multiplying or dividing, the number of significant digits in the answer is the
same as that of the factor with the least significant figures. When adding or
subtracting, the number of decimal places should equal the least number of
decimal places of any term.

\section*{Coordinate Systems}

In two dimensions, we place our mathematical description of an object's motion
within a \textbf{Cartesian coordinate system}\index{Cartesian coordinates}, also
called \emph{rectangular coordinates}.

Other times, we establish a coordinate system using \textbf{plane polar
coordinates}\index{polar coordinates}\index{plane polar coordinates}, given in
the form $(r,\theta)$. Here, $r$ is the distance from the origin to the point
with cartesian coordinates $(x, y)$ and $\theta$ is the angle between a fixed
axis and a line drawn from the origin to that point. By convention, we usually
choose the fixed axis to be positive $\hat x$ and measure $\theta$ counterclockwise
from that axis. This lets us easily switch between the two systems using
\begin{equation}
  \label{eq:xcos}
  x = r \cos \theta
\end{equation}
\begin{equation}
  \label{eq:ysin}
  y = r \sin \theta
\end{equation}
From here, we can find that
\begin{equation}
  \label{eq:tantheta}
  \tan \theta = \frac{y}{x}
\end{equation}
\begin{equation}
  \label{eq:rpyth}
  r = \sqrt{x^2 + y^2}
\end{equation}

\section*{Vector and Scalar Quantities}

A \textbf{scalar quantity}\index{scalar quantity} is completely specified by a single value with an
appropriate unit and has no direction.

A \textbf{vector quantity}\index{vector quantity} is completely specified by a number and appropriate
units plus a direction.

\section*{Unit Vectors}

A \textbf{unit vector}\index{unit vector} is a dimensionless vector having a
magnitude of exactly $1$. The unit vectors

\chapter{Newton's Laws of Motion}

The following is an excerpt from \cite[p.~20]{newton}, written by Sir Isaac Newton himself:

\begin{itemize}
  \item[\textbf{Law I.} ]
      \emph{
        Every body perserveres in its state of rest, or of uniform motion in a right
        line, unless it is compelled to change that state by forces impres'd thereon.
      }

      Projectiles persevere in their motions, so far as they are not retarded by
      the resiliance of air, or impell'd downwards by the force of gravity. A
      top, whole parts by their cohesion are preptually drown aside from
      rectilinear motions, does not cease its rotation, otherwise than it is
      retarded by air. The greater bodies of the Planets and Comets, meeting
      with less resistance in more free spaces, preserve their motions both
      progressive and circular for a much longer time.
      %\hfill\cite[p.~19]{newton}
  \item[\textbf{Law II.} ]
      \emph{
        The alteration of motion is ever proportional to the motive force
        impres'd; and is made in the direction of the right line in which that
        force is impres'd.
      }

      If any force generates a motion, a double force will generate double
      motion, a triple force triple the motion, whether that force be impres'd
      altogether and at once, or gradually and successively. And this motion
      (being always directed the same way with the generating force) if the body
      moved before, is added to or subducted from the former motion, according
      as they directly conspire with or are directly contrary to each other; or
      obliquely joyned, when they are oblique, so as to produce a new motion
      compounded from the determination of both.
      %\hfill\cite[p.~20]{newton}
  \item[\textbf{Law III.} ]
      \emph{
        To every Action there is always opposed an equal Reaction: or the mutual
        actions of two bodies upon each other are always equal, and directed to
        contrary parts.
      }

      Whatever draws or presses another is as much drawn or pressed by that
      other. If you press a stone with your finger, the finger is also pressed by
      the stone. If a horse draws a stone tyed to a rope, the horse (if I may so
      say) will be equally drawn back toward the stone: for the distended rope,
      by the same endeavour to relax or unbend it self, will draw the horse as
      much towards the stone, and will obstruct the progress of the stone as
      much as it advances that of the other. If a body impinge upon another, and
      by its force change the motion of the other; that body also (because of
      the equality of the mutual pressure) will undergo an equal change, in its
      own motion, towards the contrary part. The changes made by these actions
      are qual, not in the velocities, but in the motions of bodies; that is to
      say, if the bodies are not hinder'd by any other impediments. For because
      the motions are equally changed, the changes of the velocities made towars
      contrary parts, are reciprocally proportional to the bodies. This Law
      takes place also in Attractions, as will be proved in the next Scholium.
\end{itemize}
% \section{Kinematic Equations}
%
% \textbf{Average acceleration} is defined as the \emph{change in velocity} over
% divided by the \emph{change in time} during which the change in velocity occurs.
% Average acceleration is not normally accurate enough for our purposes, however,
% so we will speak more frequently about the \textbf{Instantaneous acceleration},
% given by
% \begin{equation}
%   \label{eq:acceleration}
%   \vec a(t) = \frac{\ud \vec v}{\ud t}
% \end{equation}
% as a function of time.
%
% For one direction, $\hat x$, we can use simple calculus to derive our basic
% kinematic equations. We can then generalize this to apply to all of kinematics.
%
% Take equation \eqref{eq:acceleration} and rewrite it for just the $\hat x$
% direction.
% \[ a_x = \frac{\ud v_x}{\ud t} \]
% Rewrite this as $\ud v_x = a_x \ud t$ and take the integral of both sides from
% $0$ to $t$, our final time.
% \[ \int \ud v_x = \int^t_0 a_x \ud t \]
% Assume acceleration is constant, giving us
% \begin{align}
%   \nonumber v_x \bigg|^t_0 &= a \int^t_0 \ud t \\
%   v_{xf} - v_{xi} &= a_x(t-0) = a_x t \\
%   \intertext{We can get an equation for velocity if we add $v_{xi}$ to each
% side:}
%   v_{xf} &= a_x t + v_{xi} \label{eq:finalvelocity}
% \end{align}
%
% Now we take the definition for \textbf{instantaneous
% velocity}\index{instantaneous velocity}
% \[ v_x = \frac{\ud x}{\ud t} \]
% and rearrange it and write it as an integral, just as we did for acceleration:
% \[ x_f - x_i = \int^t_0 v_x \ud t \]
% Now we substitute equation \ref{eq:finalvelocity} into this integral as $v_x$
% to get
% \begin{align}
%   \nonumber
%   x_f - x_i &= \int^t_0 (v_{xi} + a_x t) \ud t \\
%   \intertext{Because an integral of a sum is a sum of integrals, we can say}
%   \nonumber &= \int^t_0 v_{xi} \ud t + a_x \int^t_0 t \ud t \\
%   \nonumber &= v_{xi}(t-0)+a_x \left( \frac{t^2}{2}-0 \right) \\
%   x_f - x_i &= v_{xi} t + \frac{1}{2}a_x t^2\\
%   x_f &= \frac{1}{2}a_x t^2 + v_{xi} t + x_i
% \end{align}

%\chapter{Electric Fields}


\section{Electric Charge}

There are two kinds of electric charges, named \textbf{positive} and
\textbf{negative} by Benjamin Franklin (1706-1790)\cite[p.~643]{serway}.

Electric charge is quantized and always conserved in an isolated system.

Electrical \textbf{conductors}\index{conductors} are materials in which some of the electrons are
free electrons that are not bound to atoms and can move relatively freely
through the material; electrical \textbf{insulators}\index{insulators} are materials in which all
electrons are bound to atoms and cannot move freely through the
material.\cite[p.~644]{serway}


\section{Coulomb's Law}

\textbf{Coulomb's law}\index{Coulomb's law} describes the properties of the electric force between two
stationary charged particles. These stationary charged particles are modeled as
\textbf{point charges}\index{point charge}, theoretical charged particles of zero size. Coulomb's
law shows us that the force between two charged objects varies in
proportion with the charge $q$ placed on the objects and inversely with the
square of the distance between them.
\begin{equation}
  \label{eq:coulombslaw}
  |\vec{F}_q|=k_e\frac{q_1q_2}{r^2}
\end{equation}
Where $k_e$ is a constant called the \textbf{Coulomb constant}\index{Coulomb
constant}. In SI units, $k_e$ has the value
\begin{equation}
  \label{coulombconstant}
  k_e = 8.9876 \times 10^9 N \cdot m^2 / C^2 
\end{equation}
Often, we write the Coulomb constant in terms of the \textbf{permittivity of
free space}, $\varepsilon_0$
\begin{equation}
  k_e = \frac{1}{4 \pi \varepsilon_0}
\end{equation}
Where $\varepsilon_0$ has the value
\begin{equation}
  8.8542 \times 10^{-12} C^2 / N \cdot m^2
\end{equation}


\section{Electric Field}

The \textbf{electric field}\index{electric field} produced by a point charge is given by 
\begin{equation}
  \label{eq:electricfield}
  \vec{E} \equiv \frac{\vec{F_e}}{q_0}
\end{equation}
Where $q_0$ is a test charge, which we take by convention to be a point charge
of positive value. \textbf{An electric field exists at a point if a test charge
at that point experiences an electric force}.

Electric fields can be visualized by drawing \textbf{electric field lines},
first conceived by Faraday. The electric field vector $\vec{E}$ is tangent to
the electric field line at each point and with direction the same as the
electric field vector. The number of lines per unit area on a surface
perpendicular to the lines is proportional to the magnitude of the electric
field inthat region. The density of the field lines, therefore, represents
strength of $\vec{E}$.


\section{Motion of a Charged Particle in a Uniform Electric Field}

To get the motion of a charged particle in a uniform electrif field, simply use
$\vec{F}=ma$ to find that
\[ \vec{F_e} = q\vec{E} = m\vec{a} \]
and rearrange this to get
\begin{equation}
  \vec{a} = \frac{q\vec{E}}{m}.
\end{equation}


\section{Gauss's Law}\index{Gauss's law}


\textbf{Electric flux}\index{electric flux} is defined as the product of the
magnitude of the electric field $E$ and the surface area $A$ perpendicular to
the field.
\begin{equation}
  \Phi_E = EA
\end{equation}
Electric flux is proportional to the number of electric field lines penetrating
some surface.

In order to calculate it, we take the surface integral of $\vec E \cdot \ud \vec
A$. We are often interested in evaluating this integral across a closed surface,
where the flux would be given by
\begin{equation}
  \label{eq:electricflux}
  \Phi_E = \oint \vec E \cdot \ud \vec A = \oint E_n \ud A
\end{equation}

Gauss came up with a way to simplify this integral for \emph{closed surfaces},
often called \emph{gaussian surfaces}:
\begin{equation}
  \label{eq:gausslaw}
  \Phi_E = \oint \vec E \cdot \ud A = \frac{q_{in}}{\varepsilon_0}
\end{equation}

For a \textbf{conductor in electrostatic equilibrium}, Gauss's law tells us that
there must be zero field within it otherwise the motion of electrons would
contradict the notion of equilibrium. Immediately outside the conductor, Gauss's
law tells us
\begin{equation}
  E = \frac{\sigma}{\varepsilon_0}
\end{equation}

\section{Electric Potential}

In order to talk about potential energy in an electric field, we need to jump
through a few hoops to get there. First, remember our definition of
work\index{work}:
\[ W = \vec{F} \cdot \ud \vec s \]
In order for work to occur, we must be actually moving something. So we move a
theoretical test charge, $q_0$, from one point $a$ to another point $b$ in an
electric field, and then suddenly we are doing work. That is, we are producing a
change in potential energy for the system. This change in potential energy
$\Delta U = U_b - U_a$ is given by the integral of the work from $a$ to $b$.

So what's our force? By rearranging equation \eqref{eq:electricfield} we can
  find that $\vec{F_e} = q_0 \vec{E}$. Since this force is conservative,
Where $\Delta U$ is given by
\begin{equation}
  \Delta U = -q_0 \int_{a}^{b} \vec{E} \cdot \ud \vec{s}
\end{equation}

The \textbf{potential difference} $\Delta V$ describes a change in potential
energy. Between two points $a$ and $b$, $\Delta V = V_b - V_a$. To calculate
this value, we must take a test charge and move it from $a$ to $b$, summing the
change in potential energy in the system. The inverse of this value would give
us its potential difference.
\begin{equation}
  \label{eq:potentialdifference}
  \Delta V = \frac{\Delta U}{q_0} = -\int_{a}^{b} \vec{E} \cdot \ud \vec{s} 
\end{equation}


\chapter{Circuits}\index{circuits}

\section{Capcacitors}

A \textbf{capacitor}\index{capacitor} is a device used for \emph{storing charge.} It's different
from a battery in that it does not produce charge, only holds it.

In order to describe things like these capacitors and batteries, we are going to
use \textbf{circuit diagrams} to represent them visually in a way that provides
logical connection between the objects.

\begin{figure}[h]
  \begin{center}
    \subfigure[A capacitor.]{\boxed{
      \begin{circuitikz}
        \draw (0,0) to[C] (2,0);
      \end{circuitikz}
    }}
    \subfigure[A battery.]{\boxed{
      \begin{circuitikz}
        \draw (0,0) to[battery, i^>=$I$, l^=$-$ $+$] (2,0);
      \end{circuitikz}
    }}
    \subfigure[A resistor.]{\boxed{
      \begin{circuitikz}
        \draw (0,0) to[R, l^=$\Omega$] (2,0);
      \end{circuitikz}
    }}
  \end{center}
  \caption{Elements that make up a circuit diagram.}
  \label{fig:batterycapacitor}
\end{figure}
On a battery\index{battery}, the \emph{positive} terminal is at the \emph{higher potential} and
is represented by the \emph{longer line}.

The idea behind your most basic capacitor is that you have two plates with an
equal area, $a$, and some distance, $d$, between them. The plates are charged
with an equal but opposite charge, $Q$ and $-Q$, respectively. The air between
them serves as an insulator. Thus, there is a potential difference, $\Delta V$,
between the two plates.

We describe capacitors as having \textbf{capacitance}\index{capacitance}, the ability to store
charge. This capacitance is given by
\begin{equation}
  \label{eq:capacitance}
  C=\frac{Q}{|\Delta V|}
\end{equation}
and its units are in \textbf{Farads}\index{Farad} (F), named after Michael Faraday.
\begin{figure}[h]
  \begin{center}
  \begin{circuitikz}[scale=2.0]\draw
    (0,0) to [C, i^<=$I$, l_=$-Q$ $+Q$] (2, 0)
    (0,0) -- (0,1)
    (2,0) -- (2,1)
    (0,1) to [battery, i^>=$I$] (2,1)
    ;\end{circuitikz}
  \end{center}
  \caption{Current flow to a capacitor.}
  \label{ckt:currenttocap}
\end{figure}

It is worth noting that when analyzing circuits, current flows in the opposite
direction of the actual electron movement (shown in Figure
\ref{ckt:currenttocap}). This is purely a result of
convention; flow of positive charge would have the same effect on a circuit
as negative flow of negative charge, so this ``coordinate system'' must be
established for circuits just like we establish one for everything else.

On metals, only the electrons may move charge through a ``wire.'' However, other materials allow the
\textbf{charge carrier} to be \emph{protons} instead of electrons\footnote{or
both protons and electrons, in either direction at once!}, and if these
protons are moving in the positive direction then the current vector would point
in the same direction of the movement of charge carriers. \index{current, direction of}

For capacitors in a parallel circuit, the individual potential differences
(voltages) are equal. This is because of the layout of the physical objects:
physically (i.e. on a circuit board), a battery has opposite poles on each end.
Capacitors, by contrast, have both poles on one side. So on a parallel circuit,
the negative terminals on the capacitors are both on the same side as the
negative terminal of the battery. \emph{Ergo}, they have the same voltages.
\begin{equation}
  \Delta V_1 = \Delta V_2 = \Delta V
  \label{eq:potentialcapacitor}
\end{equation}
\begin{figure}[h]
  \begin{center}
    \subfigure[The voltages on the capacitors are the same.]{
      \begin{circuitikz}[scale=1.3]\draw
        (0,0) to [C, v^<=$\Delta V$] (2, 0)
        (0,0) -- (0,4)
        (2,0) -- (2,4)
        (0,2) to [C, v^<=$\Delta V$] (2,2)
        (0,4) to [battery, v^>=$\Delta V$] (2,4)
        ;\end{circuitikz}
      }
      \subfigure[The charges on the capacitors differ.]{
      \begin{circuitikz}[scale=1.3]\draw
        (0,0) to [C, l^=$Q_2$] (2, 0)
        (0,0) -- (0,4)
        (2,0) -- (2,4)
        (0,2) to [C, l^=$Q_1$] (2,2)
        (0,4) to [battery, v^>=$\Delta V$] (2,4)
        ;\end{circuitikz}
      }
  \end{center}
  \label{ckt:parcap}
\end{figure}
The \textbf{total charge on capacitors connected in parallel is the sum of the
charges on the individual capacitors}\cite[p.~728]{serway}.
\begin{equation}
  Q_{tot}=Q_1+Q_2
  \label{eq:capparcharge}
\end{equation}

\subsection{Equivalent Capacitors in Parallel}\index{capacitors in parallel}

Assume that in, say, Figure \ref{ckt:parcap} we want to replace $C_1$ and $C_2$
by one equivalent capacitor, $C_{eq}$. We found out in Equation
\ref{eq:potentialcapacitor} that the voltage is equal for each capacitor.
Therefore, for the equivalent capacitor,
\begin{align*}
  &Q_{tot}=C_{eq} \Delta V \\
  \intertext{Using equation \ref{eq:capparcharge}, we know that
  $Q_{tot}=Q_1+Q_2$.}
  &Q_1+Q_2=C_{eq} \Delta V \\
  \intertext{From out definition for capacitance, equation \ref{eq:capacitance},
we find that $Q_n = C_n | \Delta V_n |$.}
  &C_1 | \Delta V_1 | + C_2 | \Delta V_2 |  = C_{eq} \Delta V\\
  \intertext{But since $\Delta V$ is equivalent across all the capacitors,}
  &C_{eq}=C_1 + C_2
\end{align*}
We can generalize this for capacitors in parallel:
\begin{equation}
  C_{eq} = C_1 + C_2 + C_3 + \cdots
  \label{eq:eqcappar}
\end{equation}
This equivalent capacitance of a parallel combination of capacitors is the
algebraic sum of the individual capacitances and greater than any of the
individual capacitances.\cite[p.~728]{serway}

\subsection{Equivalent Capacitors in Series}\index{capacitors in series}

In series, \textbf{the charges on capacitors connected in series are the same:}
\begin{equation}
  \label{eq:eqcapcharser}
  Q_1 = Q_2 = Q
\end{equation}
Voltage, therefore, would be
\begin{equation}
  \label{eq:eqcapvoltser}
  \Delta V_{tot}=\Delta V_1 + \Delta V_2
\end{equation}
The \textbf{total potential difference across any number of capacitors connected
in series is the sum of the potential differences across the individual
capacitors}.\cite[p.~729]{serway}
\begin{figure}[h]
  \begin{center}
    \subfigure[The voltages on the capacitors vary.]{
      \begin{circuitikz}[scale=1.3]\draw
        (0,0) to [C, v^<=$\Delta V_2$] (4,0)
        (0,2) to [C, v^<=$\Delta V_1$] (4,2)
        (4,0) to [battery, v^>=$\Delta V$] (4,2)
        (0,0) -- (0,2)
        ;\end{circuitikz}
    }
    \subfigure[The charges on the capacitors are the same.]{
      \begin{circuitikz}[scale=1.3]\draw
        (0,0) to [C, l^=$Q$] (4,0)
        (0,2) to [C, l^=$Q$] (4,2)
        (4,0) to [battery, v^>=$\Delta V$] (4,2)
        (0,0) -- (0,2)
        ;\end{circuitikz}
    }
    \end{center}
  \label{ckt:sercap}
\end{figure}
If we recognize that \[\Delta V_{tot}=\frac{Q}{C{eq}} \] we can apply this to
our series circuit to find that
\[\frac{Q}{C_{eq}}=\frac{Q}{C_1}+ \frac{Q}{C_2}\]
Which we can generalize to find that \textbf{the inverse of the equivalent
capacitance is the algebraic sum of the inverses of the individual capacitances
and the equivalent capacitance of a series is always less than any individual
capacitance in the combination}.\cite[p.~730]{serway}

% \subsection{Spherical Capcaitors}\index{spherical capacitor}
%
% A spherical capacitor puts a charge, say $Q$, on an internal sphere and another
% equal but opposite charge, $-Q$, on the outside sphere. The electric potential,
% $\Delta V$, is given by
%
% \begin{align*}
%   \Delta V &= \int_a^b \vec{E} \cdot \ud \vec{S} \\
%   \intertext{Where $\ud \vec{S}$ is a very small chunk of the sphere. We don't
%   know how to integrate this. But we can change it into something we know how to
%   integrate by replacing $\vec{E}$ with $\vec{E}$ as a function of the radius,
%   $r$.}
%   \Delta V &= - \int_{r_1}^{r_2} \vec{E}(r) \ud r \\
%   \Delta V &= - \int_{r_1}^{r_2} \frac{k_e Q}{r^2} \ud r \\
%   \intertext{Constants can be removed from the integral:}
%   \Delta V &= -k_e Q \int_{r_1}^{r_2} \frac{\ud r}{r^2} \\
%   \Delta V &= -k_e Q \left( \frac{1}{r_2}-\frac{1}{r_1}\right) \\
%   \intertext{Now, replacing this $\Delta V$ into equation \eqref{eq:capacitance}, we
% find that}
% C&=\frac{Q}{k_e Q\left( \frac{1}{r_2}-\frac{1}{r_1}\right)}
% \end{align*}
% Where $k_e$ is the \textbf{Coulomb constant}\index{Coulomb constant}, given by
% \[ k_e = \frac{1}{4 \pi \varepsilon_0} \approx 8.987551 \times 10^{19} N \cdot
% m^2 / C^2 \]
% Thus, we find that the capacitance of a spherical conductor is given by:
% \begin{equation}
%   C_{sphere}=\frac{4 \pi \varepsilon_0}{(\frac{1}{r_1}-\frac{1}{r_2})}
%   \label{eq:spherecapacitance}
% \end{equation}
% Where $r_1$ is the radius of the inner sphere, and $r_2$ is the radius of the
% outer sphere.
%
% % \begin{ex}
% %   Say we would like to treat the Earth as a capacitor. What would its
% %   capacitance be?
% %   \begin{sol}
% %     Use equation \eqref{eq:spherecapacitance}. Since outer space essentially
% %     serves as the outer limit for our capacitor, $b\to\infty$ and our
% %     capacitance becomes
% %     \[ C_{\text{earth}}=\frac{4 \pi \varepsilon_0}{\frac{1}{r_{\text{earth}}}}=4 \pi
% %       \varepsilon_0 r_{\text{earth}} \]
% %     Which is approximately only
% %     \[ 0.7 mF \]
% %     Farads are \emph{very} large.
% %   \end{sol}
% % \end{ex}
% % dr gore
% %\begin{ex}
% %  Consider a conductor formed by an inner cylinder with a radius $r_1$ and an
% %  outer cylinder with radius $r_2$. Its length, $l$, is much greather than $r_1$
% %  and $r_2$, which allows us to neglect end behavior. This describes a standard coaxial cable.
% %  %From Gauss's Law, we know that our electric field at any point is given by $E
% %  %= 2 k_e \lambda / r$, and thus our electric potential would be found by
% %  % \[\Delta V = -2 k_e \lambda \ln{|r_2/r_1|} \]
% %  % To find the capacitance, substitute the absolute value of $\Delta V$ into
% %  % equation \eqref{eq:capacitance}:
% %  % \[ C=\frac{Q}{\Delta V} = \frac{Q}{|V_2-V_1|} = \frac{r_1r_2}{k_e(r_2-r_1)} \]
% %\end{ex}
%
% \appendix


