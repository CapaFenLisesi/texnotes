\chapter{Electric Fields}


\section{Electric Charge}

There are two kinds of electric charges, named \textbf{positive} and
\textbf{negative} by Benjamin Franklin (1706-1790)\cite[p.~643]{serway}.

Electric charge is quantized and always conserved in an isolated system.

Electrical \textbf{conductors}\index{conductors} are materials in which some of the electrons are
free electrons that are not bound to atoms and can move relatively freely
through the material; electrical \textbf{insulators}\index{insulators} are materials in which all
electrons are bound to atoms and cannot move freely through the
material.\cite[p.~644]{serway}


\section{Coulomb's Law}

\textbf{Coulomb's law}\index{Coulomb's law} describes the properties of the electric force between two
stationary charged particles. These stationary charged particles are modeled as
\textbf{point charges}\index{point charge}, theoretical charged particles of zero size. Coulomb's
law shows us that the force between two charged objects varies in
proportion with the charge $q$ placed on the objects and inversely with the
square of the distance between them.
\begin{equation}
  \label{eq:coulombslaw}
  |\vec{F}_q|=k_e\frac{q_1q_2}{r^2}
\end{equation}
Where $k_e$ is a constant called the \textbf{Coulomb constant}\index{Coulomb
constant}. In SI units, $k_e$ has the value
\begin{equation}
  \label{coulombconstant}
  k_e = 8.9876 \times 10^9 N \cdot m^2 / C^2 
\end{equation}
Often, we write the Coulomb constant in terms of the \textbf{permittivity of
free space}, $\varepsilon_0$
\begin{equation}
  k_e = \frac{1}{4 \pi \varepsilon_0}
\end{equation}
Where $\varepsilon_0$ has the value
\begin{equation}
  8.8542 \times 10^{-12} C^2 / N \cdot m^2
\end{equation}


\section{Electric Field}

The \textbf{electric field}\index{electric field} produced by a point charge is given by 
\begin{equation}
  \label{eq:electricfield}
  \vec{E} \equiv \frac{\vec{F_e}}{q_0}
\end{equation}
Where $q_0$ is a test charge, which we take by convention to be a point charge
of positive value. \textbf{An electric field exists at a point if a test charge
at that point experiences an electric force}.

Electric fields can be visualized by drawing \textbf{electric field lines},
first conceived by Faraday. The electric field vector $\vec{E}$ is tangent to
the electric field line at each point and with direction the same as the
electric field vector. The number of lines per unit area on a surface
perpendicular to the lines is proportional to the magnitude of the electric
field inthat region. The density of the field lines, therefore, represents
strength of $\vec{E}$.


\section{Motion of a Charged Particle in a Uniform Electric Field}

To get the motion of a charged particle in a uniform electrif field, simply use
$\vec{F}=ma$ to find that
\[ \vec{F_e} = q\vec{E} = m\vec{a} \]
and rearrange this to get
\begin{equation}
  \vec{a} = \frac{q\vec{E}}{m}.
\end{equation}


\section{Gauss's Law}\index{Gauss's law}


\textbf{Electric flux}\index{electric flux} is defined as the product of the
magnitude of the electric field $E$ and the surface area $A$ perpendicular to
the field.
\begin{equation}
  \Phi_E = EA
\end{equation}
Electric flux is proportional to the number of electric field lines penetrating
some surface.

In order to calculate it, we take the surface integral of $\vec E \cdot \ud \vec
A$. We are often interested in evaluating this integral across a closed surface,
where the flux would be given by
\begin{equation}
  \label{eq:electricflux}
  \Phi_E = \oint \vec E \cdot \ud \vec A = \oint E_n \ud A
\end{equation}

Gauss came up with a way to simplify this integral for \emph{closed surfaces},
often called \emph{gaussian surfaces}:
\begin{equation}
  \label{eq:gausslaw}
  \Phi_E = \oint \vec E \cdot \ud A = \frac{q_{in}}{\varepsilon_0}
\end{equation}

For a \textbf{conductor in electrostatic equilibrium}, Gauss's law tells us that
there must be zero field within it otherwise the motion of electrons would
contradict the notion of equilibrium. Immediately outside the conductor, Gauss's
law tells us
\begin{equation}
  E = \frac{\sigma}{\varepsilon_0}
\end{equation}

\section{Electric Potential}

In order to talk about potential energy in an electric field, we need to jump
through a few hoops to get there. First, remember our definition of
work\index{work}:
\[ W = \vec{F} \cdot \ud \vec s \]
In order for work to occur, we must be actually moving something. So we move a
theoretical test charge, $q_0$, from one point $a$ to another point $b$ in an
electric field, and then suddenly we are doing work. That is, we are producing a
change in potential energy for the system. This change in potential energy
$\Delta U = U_b - U_a$ is given by the integral of the work from $a$ to $b$.

So what's our force? By rearranging equation \eqref{eq:electricfield} we can
  find that $\vec{F_e} = q_0 \vec{E}$. Since this force is conservative,
Where $\Delta U$ is given by
\begin{equation}
  \Delta U = -q_0 \int_{a}^{b} \vec{E} \cdot \ud \vec{s}
\end{equation}

The \textbf{potential difference} $\Delta V$ describes a change in potential
energy. Between two points $a$ and $b$, $\Delta V = V_b - V_a$. To calculate
this value, we must take a test charge and move it from $a$ to $b$, summing the
change in potential energy in the system. The inverse of this value would give
us its potential difference.
\begin{equation}
  \label{eq:potentialdifference}
  \Delta V = \frac{\Delta U}{q_0} = -\int_{a}^{b} \vec{E} \cdot \ud \vec{s} 
\end{equation}

