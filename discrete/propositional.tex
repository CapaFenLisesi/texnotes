\chapter{Propositional Logic}
\epigraph{This is your last chance. After this, there is no turning back. You
take the blue pill---the story ends, you wake up in your bed and believe
whatever you want to believe. You take the red pill---you stay in Wonderland and
I show you how deep the rabbit-hole goes.}{Morpheus, \emph{The Matrix}, 1999}
\label{ch:propositional}

The whole idea of propositional logic is that statements in mathematics have truth value and there is a logical process that we use to evaluate that truth value.
Perhaps most importantly, this chapter will introduce us to \emph{DeMorgan's Laws}, which are relevant everywhere from computer science (in simplifying boolean operators) to the writings of Aristotle.

Propositional logic is also sometimes called \textbf{propositional calculus},
\index{propositional calculus}
from the latin \emph{calculus} meaning ``pebble,'' since in early days pebbles
were used for counting.

\section{Introduction to Propositional Logic}
\label{sec:propintro}

\index{proposition}
A \textbf{proposition}\index{proposition} just a sentence that makes a statement.
\begin{ex}
  ``My house is red.''
\end{ex}

\index{propositional calculus}
  \textbf{Propositional calculus} (or \emph{propositional logic}) assigns propositions \emph{propositional variables}
  (e.g. \(p, q, r, s, \ldots\)) and deals with their logic and \emph{truth value}.
  \begin{ex}
    Let $p$ be ``my house is red.''
  \end{ex}

\index{negation}
  Let \(p\) be a proposition. The \textbf{negation} of \(p\), denoted by \[\neg p\] is the statement
  ``it is not the case that \(p\).''
  The truth value of \(\neg p\) is the opposite of the truth value of \(p\).
  \begin{ex}
    Let $p$ be ``my house is red.''

    Then $\neg p$ is ``it is not the case that my house is red,'' or simply ``my house is not red.''
  \end{ex}

\index{conjunction}
  Let \(p\) and \(q\) be propositions. The \textbf{conjunction} of \(p\) and \(q\), denoted by \[ p \wedge q \] is the proposition
  ``\(p\) and \(q\).'' It is true when both \(p\) and \(q\) are true and false otherwise.
  \begin{ex}
    Let $p$ be ``Kevin likes Sarah'' and let $q$ be ``Sarah likes Kevin.'

    Then $p \wedge q$ is ``Kevin likes Sarah and Sarah likes Kevin,'' or ``Kevin and Sarah like each other.''
    This statement would be false if either of the two did not like the other.
  \end{ex}

\index{disjunction}
  Let \(p\) and \(q\) be propositions. The \textbf{disjunction} of \(p\) and \(q\), denoted by \[ p \vee q \] is the proposition
  ``\(p\) and \(q\).'' It is true when either \(p\) or \(q\) are true and false otherwise.
  \begin{ex}
    Let $p$ denote ``Kevin hates bagels'' and $q$ denote ``Kevin hates poppy seeds.''

    The proposition $p \vee q$ is the statement ``Kevin hates bagels or poppy seeds.''
    Note, here, that there is an implicit cue in the language that Kevin could, indeed, hate both bagels and poppy seeds.
    The statement would be true if he were distasteful toward either one of them, or both.
  \end{ex}

\index{exclusive or}
  The \textbf{exclusive or} of \(p\) and \(q\), denoted by \[p \oplus q\] is true when exactly one of \(p\) and \(q\) is true and false otherwise.
  \begin{ex}
    Let $p$ mean ``I could sleep in all day today'' and $q$ mean ``I could go to work.''

    The statement $p \oplus q$ is true when only one of $p$ or $q$ is true.
    It is the sentence ``I could go to work, or I could sleep in all day today.''
    Since it does not make sense to do both at once, the \emph{exclusive or} is implied in our use of the English language.

    We should keep an eye out for this distinction between disjunciton and exclusive or, as it requires careful
    attention to the subtleties in our use of language.
  \end{ex}

  \index{conditional statement}
  The \textbf{conditional statement} \[p \implies q\] is the propsition ``if p, then q.''
  The conditional statement \(p \implies q\) is false when \(p\) is true and \(q\) is false, and true otherwise.
  Other ways of writing the conditional statement can be found in \tabref{tab:conditionals}.
  \index{hypothesis}\index{antecedent}\index{premise}
  \(p\) is called the \textbf{hypothesis} (or antecedent or premise).
  \index{conclusion}\index{consequence}
  \(q\) is called the \textbf{conclusion} (or consequence).
  \begin{ex}
    Let $p$ be ``Joe broke his arm'' and $q$ be ``Joe will go to the hospital.''

    The statement $p \implies q$ is the statement ``If Joe broke his arm, then he will go to the hospital.''
    It is true if Joe only goes to the hospital, if he only breaks his arm, and if both events occur.
    The only situation in which it is false is if Joe breaks his arm, then does not go to the hospital.
  \end{ex}
  \begin{table}[H]
  \centering
  %\boxed{
    \begin{tabular}{p{2in} p{2in}}
      ``if \(p\), then \(q\)'' & ``\(p\) implies \(q\)'' \\
      ``if \(p\), \(q\)'' & ``\(p\) only if \(q\)'' \\
      ``\(p\) is sufficient for \(q\)'' & ``a sufficient condition for \(q\) is \(p\)'' \\
      ``\(q\) if \(p\)'' & ``\(q\) whenever \(p\)`` \\
      ``\(q\) when \(p\)'' & ``\(q\) is necessary for \(p\)'' \\
      ``a necessary condition for \(p\) is \(q\)'' & ``\(q\) follows from \(p\)'' \\
      ``\(q\) unless \(p\)''
    \end{tabular}
  %}
  \caption{Other ways of writing the conditional statement \(p \implies q\).}
  \label{tab:conditionals}
\end{table}
\index{converse}
  The \textbf{converse} of \(p \implies q\) is \(q \implies p\).

\index{contrapositive}\label{def:contrapositive}
  The \textbf{contrapositive} of
  \(p \implies q\)
  is
  \(\neg q \implies \neg p\).
  It is logically equivalent to
  \(p \implies q\).
  We can prove this with the following \emph{truth table}:
  \begin{table}[H]
    \centering
    \begin{tabular}{|>$c<$|>$c<$|>$c<$|>$c<$|}
      \hline
      p & q & p\implies q & \neg q \implies \neg p\\
      \hline
      0 & 0 & 1 & 1 \\
      0 & 1 & 1 & 1 \\
      1 & 0 & 0 & 0 \\
      1 & 1 & 1 & 1 \\\hline
    \end{tabular}
    \caption{A truth table for $p\implies q$ and $\neg q \implies \neg p$.}
    \label{tab:contrapositive}
  \end{table}
  A \textbf{truth table} can be used to demonstrate the logical relationships between statements.

  This property of contrapositives is especially useful when we are trying to prove a statement,
  as its contrapositive is often easier to prove than the statement itself.
  This subject is detailed further in \secref{sec:contrapositive}.

\index{inverse}
  The \textbf{inverse} of \(p \implies q\) is \(\neg p \implies \neg q\).
  It has the opposite truth value of the original statement.

  \index{biconditional statement}
  The \textbf{biconditional statement}, also called \emph{bi-implication}, for $p$ and $q$ is \[p \iff q.\] This is the statement
  ``\(p\) if and only if \(q\).''
  It is true when both \(p\) and \(q\) have the same truth value, and false otherwise.
\begin{table}[h]
  \centering
    \begin{tabular}{l}
      ``\(p\) is necessary and sufficient for \(q\)'' \\
      ``if \(p\), then \(q\), conversely'' \\
      ``p iff q''
    \end{tabular}
  \label{tab:biconditionals}
\end{table}
In mathematical theorems, definitions, and related elements later in this text, we will often see the biconditional operator written \emph{iff}.

\section{Logical Equivalence}
\index{tautology}
  A \textbf{tautology} is a compound proposition that is always true, regardless of the truth values of the variables that occur in it.
\index{contradiction}
  A \textbf{contradiciton} is a compound proposition that is always false.
\index{contingency}
  A \textbf{contingency} is a compound proposition that is neither a tautology nor a contradiction.
  It can be either true or false.
\begin{ex}
  \( p \iff q \) is logically equivalent to \( (p \implies q) \wedge (q \implies p)\).
  We state this by writing
  \[ p \leftrightarrow q \equiv (p \implies q) \wedge (q \implies p).\]
\end{ex}
\begin{table}
  \centering
    \begin{tabular}{>{\(}l<{\)} >{\(}r<{\)}|>{\(}c<{\)}|>\(c<\)|>\(c<\)|>\(c<\)}
      p & q & \neg q & p \lor \neg q & p \land q & (p \lor \neg q) \implies (p \land q) \\ \hline
      1 & 1 & 0      & 1             & 1         & 1 \\
      1 & 0 & 1      & 1             & 0         & 0 \\
      0 & 1 & 0      & 0             & 0         & 1 \\
      0 & 0 & 1      & 1             & 0         & 0
    \end{tabular}
  \caption{The truth table for $(p \lor \neg q) \implies (p \land q)$}
\end{table}
\index{logical equivalence}
The compound propositions \(p\) and \(q\) are \textbf{logically equivalent} if \(p \iff q\) is a tautology.
The notation \(p \equiv q\) denotes that \(p\) and \(q\) are logically equivalent.
\begin{remark}
  The symbol \(\equiv\) is not a logical connection, and \(p\equiv q\) is not a compound proposition,
  but rather it is the statement that \(p \iff q\) is a tautology.
  The symbol \(\iff\) is sometimes used instead of \(\equiv\) to denote logical equivalence.
\end{remark}

\section{Precedence}
For more complex propositions, we need rules to tell which logical operators come first when we read them.

The \emph{not} operator takes the highest precedence.
Then conjunction, followed by disjunction.
Then conditional, and finally bi-conditional operators.
\begin{table}[H]
  \centering
  %\boxed{
    \begin{tabular}{r|cl}
      1 & $\neg$      &not\\
      2 & $\land$     &and \\
      3 & $\lor$      & or \\
      4 & $\implies$  & conditional \\
      5 & $\iff$      &biconditional
    \end{tabular}
  %}
  \caption{Logical operator precedence.}
  \label{tab:precedence}
\end{table}
\begin{ex}
  Is the statement \[ p \land q \lor r \] equivalent to the statement
  \( ( p \land q ) \lor r\)
  or \( p \land (q \lor r)\)?
  \begin{sol}
    Let's look at \tabref{tab:precedence}.
    It states that $\land$ (``and'') operators come before $\lor$ (``or'') operators.
    Therefore,
    \[ p \land q \lor r \equiv (p \land q) \lor r \text{.}\]
  \end{sol}
\end{ex}

\section{DeMorgan's Laws}
\textbf{DeMorgan's Laws} are an important concept in propositional logic and boolean algebra.

\begin{equation}
  \neg (p \land q) \equiv \neg p \lor \neg q
\end{equation}
\begin{equation}
  \neg(p \lor q) \equiv \neg p \land \neg q
\end{equation}
These can be proven using a \textbf{truth table}.
In order to construct a truth table, we must display all possible values of the propositional variables,
and the corresponding values of the propositional statement for each combination.
Intermediary steps may aid one's understanding, though they are not necessary in the final result.
\begin{table}[H]
  \centering
  \boxed{
    \begin{tabular}{>\( l <\) >\(r<\)|>\(c<\)|>\(c<\)}
      p & q & \neg(p \land q) & \neg p \lor \neg q \\ \hline
      0 & 0 & 1 & 1 \\
      0 & 1 & 1 & 1\\
      1 & 0 & 1 & 1\\
      1 & 1 & 0 & 0\\
    \end{tabular}
  }
  \caption{A proof of DeMorgan's first law.}
\end{table}
\begin{table}[H]
  \centering
  \boxed{\
    \begin{tabular}{>\( l <\) >\(r<\)|>\(c<\)|>\(c<\)}
      p & q & \neg(p \lor q) & \neg p \land \neg q \\ \hline
      0 & 0 & 1 & 1 \\
      0 & 1 & 0 & 0 \\
      1 & 0 & 0 & 0 \\
      1 & 1 & 0 & 0 \\
    \end{tabular}
  }
  \caption{A proof of DeMorgan's second law.}
\end{table}

Extending DeMorgan's laws by the association laws for disjunction and conjunction, shown in \tabref{tab:logequiv}, they become:
\begin{equation}
 \neg\left(\bigwedge^n_{n=1} p_n\right)=\bigvee^n_{n=1} \neg p_n
\end{equation}
\begin{equation}
 \neg\left(\bigvee^n_{n=1} p_n\right)=\bigwedge^n_{n=1} \neg p_n
\end{equation}
Using DeMorgan's Laws, we can \emph{negate conjunctions and disjunctions}.
In computer science, we use DeMorgan's Laws to simplify boolean expressions.

\begin{ex}
  Negate the following statement:
  ``Miguel has a cellphone and he has a laptop.''
  \begin{sol}
    Let \(p\) be ``Miguel has a cell phone.''
    Let \(q\) be ``Miguel has a laptop.''
    The negation of $p \land q$ is
    \[ \neg p \lor \neg q, \]
    which means ``Miguel does not have a cell phone or he does not have a laptop.''
  \end{sol}
\end{ex}

\section{Useful Logical Equivalences}

In a proof or simplification of propositional logic statements, propositions like $p \implies q$
are difficult for us to work with.
We have very few laws or equivalences which work directly with them, so we often must convert them into an equivalent form using other operators.

We can convert $p\implies q$ to a proposition using only the $\neg$ and $\lor$ operators using the logical equivalence
\begin{equation}
  p \implies q \equiv \neg p \lor q,
\end{equation}
which we will prove using a truth table in \tabref{tab:conditionalproof}.
\begin{table}[H]
  \centering
  \boxed{\
    \begin{tabular}{>\( l <\) >\(r<\)|>\(c<\)|>\(c<\)}
      p & q & p \implies q & \neg p \lor q \\ \hline
      1 & 1 & 1 & 1 \\
      1 & 0 & 0 & 0 \\
      0 & 1 & 1 & 1 \\
      0 & 0 & 1 & 1 \\
    \end{tabular}
  }
  \caption{A proof of \(p \implies q \equiv \neg p \lor q\).}
  \label{tab:conditionalproof}
\end{table}
\begin{table}[H]
  \centering
    \begin{tabular}{>\(l<\) r}
      \textbf{Proposition} & \textbf{Name} \\ \hline\noalign{\smallskip}
      p \land T \equiv p & \multirow{2}{*}{identity laws} \\
      p \lor f \equiv p \\\hline
      p \lor T \equiv T & \multirow{2}{*}{domination laws} \\
      p \land F \equiv F \\\hline
      p \lor p \equiv p & \multirow{2}{*}{idempotent laws} \\
      p \land p \equiv p \\\hline\noalign{\smallskip}
      \neg (\neg p) & double negation law \\\noalign{\smallskip}\hline
      p \lor q \equiv q \lor p & \multirow{2}{*}{commutative laws} \\
      p \land q \equiv q \land p \\\hline
      (p \lor q) \lor r \equiv p \lor (q \lor r) & \multirow{2}{*}{associative laws} \\
      (p \land q) \land r \equiv p \land (q \land r) \\\hline
      p \land (q \lor r) \equiv (p \land q) \lor (p \land r) & \multirow{2}{*}{distributive laws} \\
      p \lor (q \land r) \equiv (p \lor q) \land (p \lor r) \\\hline
      \neg (p \lor q) \equiv \neg p \lor \neg q & \multirow{2}{*}{DeMorgan's laws} \\
      \neg (p \lor q) \equiv \neg p \land \neg q \\\hline
      p \lor (p \land q) \equiv p & \multirow{2}{*}{absorbtion laws} \\
      p \land (p \lor q) \equiv p \\\hline
      p \lor \neg p \equiv T & \multirow{2}{*}{negation laws} \\
      p \land \neg p \equiv F
    \end{tabular}
  \caption{Useful logical equivalence laws.}
  \label{tab:logequiv}
\end{table}

\section{Proving Logical Equivalences}

We can prove logical equivalences by using the rules in \tabref{tab:logequiv}, and showing
how one step leads to another until we reach something we know to be true.
Alternatively, we could start with a statement that we know to be true and work our way to the logical equivalence we are trying to prove.
Both of these proof methods are valid, though not especially rigorous, as they rely upon rules that we may not have established the truth value of with the mathematical rigor required to consider them \emph{formal proofs}.
\begin{ex}
  Show that \(\neg (p \implies q)\) and \( p \land \neg q\) are logically equivalent.
  \begin{sol}
    \[
      \begin{fitch}
        \fb \neg (p \implies q) \equiv \neg (p \implies q) \\
        \fa \neg(p \implies q) \equiv \neg(\neg p \lor q) & \(p \implies q \equiv \neg p \lor q\) \\
        \fa \neg(p \implies q) \equiv \neg(\neg p) \land \neg q & \text{DeMorgan's Law} \\
        \fa \neg(p \implies q) \equiv p \land \neg q & \text{Double negation}
      \end{fitch}
    \]
  \end{sol}
\end{ex}

Alternatively, they can be proven using truth tables, as described in \secref{def:contrapositive}.

% \section{Bit Operations}
%
% These logical operators can also be performed on \textbf{bits} (0 is false, 1 is true).
% These are called \textbf{bit operations}, and can even be performed on \textbf{bit strings}, sequences of zero or more bits.
% The \textbf{length} of a bit string is the number of bits in the string.
% \begin{ex}
%   Perform bitwise OR, AND, and XOR operations on the following bit strings.
%   \begin{align*}
%     01 \, 1011 \, 0100 & \\
%     11 \, 0001 \, 1101 &
%   \end{align*}
%   \begin{sol}
%     \begin{align*}
%       01 \, 1011 \, 0100 & \\
%       11 \, 0001 \, 1101 &
%       \\ \\
%       11 \, 1011 \, 1111 & \quad \text{bitwise OR} \\
%       01 \, 0001 \, 0100 & \quad \text{bitwise AND} \\
%       10 \, 1010 \, 1011 & \quad \text{bitwise XOR}
%     \end{align*}
%   \end{sol}
% \end{ex}

% \section{Examples}
%
% % bad example, copied from a textbook.
% % I need to write my own.
% \begin{ex}\cite[p.~14]{rosen}
%   Determine whether each of these conditional statements is true or false.
%   \begin{itemize}
%     \item[a) ] if \(1+1=2\), then \(2+2=5\)
%     \item[b) ] if \(1+1=3\), then \(2+2=4\)
%     \item[c) ] if \(1+1=3\), then \(2+2=5\)
%     \item[d) ] if monkeys can fly, then \(1+1=3\)
%   \end{itemize}
%   \begin{sol}
%     In each case, we simply determine the truth value of the hypothesis and the conclusion, then use the definition of the truth value of conditional statements to get our answer.
%     \begin{itemize}
%       \item[a) ] The hypothesis is true, and the conclusion is false, so the statement is false.
%       \item[b) ] Since the hypothesis is false and the conclusion is true, the statement is true.
%       \item[c) ] Since the hypothesis is false, regardless of the conclusion the conditional statement must be true.
%       \item[d) ] Since the hypothesis is false, the conditional is true.
%     \end{itemize}
%   \end{sol}
% \end{ex}

\section{Propositional Satisfiability}
% I'm pretty sure this section is ripped near-straight from a textbook too.
% I really need to write my own.

A compound proposition is \textbf{satisfiable} if there is an assignment of truth values to its variables that make it true.
When no such assignments exists, it is \textbf{unsatisfiable}.
A particular assignment of truth values that shows a compound proposition satisfiable is a \textbf{solution}.

To show a compound proposition is satisfiable, we need only demonstrate one possible solution.
However, to demonstrate it unsatisfiable, we would need to show every possible assignment of truth values and show why they make it false.
Thus, reasoning is often more useful than truth tables in showing a compound proposition is unsatisfiable.

% \begin{ex}
%   Determine whether each of the following is satisfiable.
%   \begin{itemize}
%     \item[a) ] \( (p \lor \neg q) \land (q \lor \neg r) \land (r \lor \neg p) \)
%     \item[b) ]  \((p \lor q \lor r) \land (\neg p \lor \neg q \lor \neg r) \)
%     \item[c) ]  \( (p \lor \neg q) \land (q \lor \neg r) \land (r \lor \neg p) \land (p \lor q \lor r) \land (\neg p \lor \neg q \lor \neg r)\)
%   \end{itemize}
%   \begin{sol}
%     (a) is true when \(p\), \(q\) and \(r\) have the same truth value.
%     (b) is true when at least one of \(p\), \(q\), or \(r\) is false and one is true.
%     So (c), a combination of (a) and (b), is unsatisfiable.
%   \end{sol}
% \end{exK
