\chapter{Propositional Logic}
\epigraph{This is your last chance. After this, there is no turning back. You
take the blue pill---the story ends, you wake up in your bed and believe
whatever you want to believe. You take the red pill---you stay in Wonderland and
I show you how deep the rabbit-hole goes.}{Morpheus, \emph{The Matrix}, 1999}
\label{ch:propositional}

The whole idea of propositional logic is that statements in mathematics have truth value and there is a logical process that we use to evaluate that truth value.
Perhaps most importantly, this chapter will introduce us to \emph{DeMorgan's Laws}, which are relevant everywhere from computer science (in simplifying boolean operators) to the writings of Aristotle.

\begin{remark}
Propositional logic is also sometimes called \textbf{propositional calculus},
\index{propositional calculus}
from the Latin \emph{calculus} meaning ``pebble'', since in early days pebbles
were used for counting.\cite{wiktionary-calculus}
\end{remark}

\section{Introduction to Propositional Logic}
\label{sec:propintro}

\begin{defn}[proposition]
A \definition{proposition}{proposition} a sentence that makes a statement
and has a truth value.
\end{defn}

\begin{ex}
  The following sentence is a proposition:
  \begin{equation}
    \text{``My house is red.''}
    \label{eq:ex:proposition}
  \end{equation}
  The way we know that \eref{eq:ex:proposition} is a proposition is because
  it has a theoretically knowable truth value.
  We will write that \eref{eq:ex:proposition} is \ltrue{} when its author's
  house is, indeed, red---and that \eref{eq:ex:proposition} is \lfalse{}
  if that is not the case.
\end{ex}

  \definition{Propositional calculus}{propositional calculus}
  (or \altdefinition{propositional logic}{propositional logic})
  assigns propositions \definition{propositional variables}{propositional variables}
  (e.g. \(p, q, r, s, \ldots\))
  and deals with their logic and \emph{truth value}.
  \begin{ex}
    Let $p$ be ``my house is red.''
  \end{ex}

\begin{defn}[negation]
  Let \(p\) be a proposition.
  The \definition{negation}{negation} of \(p\),
  $\neg p$,
  is the statement
  ``it is not the case that \(p\).''
  The truth value of \(\neg p\) is the opposite of the truth value of \(p\).
\end{defn}

\begin{ex}
  Let $p$ be ``my house is red.''
  Then $\neg p$ is
  ``it is not the case that my house is red,''
  or simply
  ``my house is not red.''
\end{ex}

\begin{defn}[conjunction]
  Let $p$ and $q$ be propositions.
  The \definition{conjunction}{conjunction}
  $p \wedge q$
  of $p$ and $q$,
  is the proposition
  ``$p$ and $q$.''
  The conjunction is \ltrue{} when both $p$ and $q$ are \ltrue{} and \lfalse{} otherwise.
\end{defn}

\begin{ex}
  Let $p$ be ``Kevin likes Sarah'' and let $q$ be ``Sarah likes Kevin.'
  Then $p \wedge q$ is
  ``Kevin likes Sarah and Sarah likes Kevin,''
  or ``Kevin and Sarah like each other.''
  This statement would be \lfalse{} if either of the two did not like the other.
\end{ex}

\begin{defn}[disjunction]
  Let $p$ and $q$ be propositions.
  The \definition{disjunction}{disjunction}
  of $p$ and $q$,
  written $p \vee q$,
  is the proposition
  ``$p$ or $q$.''
  It is \ltrue{} when either $p$ or $q$ are \ltrue{} and \lfalse{} otherwise.
\end{defn}

\begin{ex}
  Let $p$ denote ``Kevin hates bagels'' and
  $q$ be the assertion that ``Kevin hates poppy seeds.''
  The proposition $p \vee q$ is the statement ``Kevin hates bagels or poppy seeds.''
  \begin{remark}
    Note, here, that there is an implicit cue in the language that Kevin could,
    indeed, hate both bagels and poppy seeds.
    The statement would be \ltrue{} if he were distasteful toward either one of them,
    or both. In \exref{ex:prop:xordefn} we will see language cues toward the
    other usage of English \emph{or}.
  \end{remark}
\end{ex}

\begin{defn}[exclusive or]
  The \definition{exclusive or}{exclusive or}
  $p \oplus q$
  of propositional variables
  $p$ and $q$ is \ltrue{} when exactly one of $p$ and $q$ is \ltrue{},
  and \lfalse{} otherwise.
\end{defn}

\begin{ex}
  Let $p$ mean ``I could sleep in all day today''
  and $q$ mean ``I could go to work.''
  Then the statement $p \oplus q$ is \ltrue{} when only one of $p$ or $q$ is \ltrue{}.
  It is the sentence ``I could go to work, or I could sleep in all day today.''
  Since it does not make sense to do both at once,
  the \definitionintro{exclusive or} is implied by our choice of words.
  \begin{remark}
    We should keep an eye out for this distinction between disjunction and
    exclusive or, as it requires careful attention to the subtleties in our use of
    language.
  \end{remark}
  \label{ex:prop:xordefn}
\end{ex}

\begin{defn}[conditional statement]
  The \definition{conditional statement}{conditional statement}
  $p \implies q$
  is the proposition ``if p, then q.''
  It is \lfalse{} when $p$ is \ltrue{} and $q$ is \lfalse{}, and \ltrue{} otherwise.
  \label{def:conditional}
\end{defn}

\begin{defn}[hypothesis]
  In a conditional statement, $p$ is called the \definition{hypothesis}{hypothesis}
  (or \altdefinition{antecedent}{antecedent} or \altdefinition{premise}{premise}).
\end{defn}

\begin{defn}[conclusion]
  In a conditional statement, the variable $q$ is called the \definition{conclusion}{conclusion}
  or \altdefinition{consequence}{consequence}.

  Other ways of writing the conditional statement are shown in \tabref{tab:conditionals}.
\end{defn}

\begin{ex}
  Let $p$ be ``Joe broke his arm''
  and let $q$ be ``Joe will go to the hospital.''

  If one were to state that $p \implies q$, they mean
  ``If Joe broke his arm, then he will go to the hospital.''
  This conditional is \ltrue{} if Joe only goes to the hospital,
  if he only breaks his arm, and if both events occur.
  The only situation in which it is \lfalse{} is if Joe breaks his arm,
  but does not go to the hospital.
\end{ex}

\begin{table}[H]
  \centering
    \begin{tabular}{p{2in} p{2in}}
      ``if \(p\), then \(q\)''                     & ``\(p\) implies \(q\)'' \\
      ``if \(p\), \(q\)''                          & ``\(p\) only if \(q\)'' \\
      ``\(p\) is sufficient for \(q\)''            & ``a sufficient condition for \(q\) is \(p\)'' \\
      ``\(q\) if \(p\)''                           & ``\(q\) whenever \(p\)`` \\
      ``\(q\) when \(p\)''                         & ``\(q\) is necessary for \(p\)'' \\
      ``a necessary condition for \(p\) is \(q\)'' & ``\(q\) follows from \(p\)'' \\
      ``\(q\) unless \(p\)''
    \end{tabular}
  \caption{Other ways of writing the conditional statement \(p \implies q\).}
  \label{tab:conditionals}
\end{table}

\begin{defn}[converse]
  The \definition{converse}{converse} of \(p \implies q\) is \(q \implies p\).
\end{defn}

\begin{defn}
  The \definition{contrapositive}{contrapositive} of
  $p \implies q$ is $\neg q \implies \neg p$.
  \label{def:contrapositive}
\end{defn}

\begin{theorem}
  Where $p$ and $q$ are propositional variables,
  the contrapositive statement
  $\neg q \implies \neg p$
  \pdefnref{def:contrapositive}
  is logically equivalent to the conditional statement
  $p \implies q$
  \pdefnref{def:conditional}.
  \label{thm:contrapositive-conditional}
\end{theorem}

\thref{thm:contrapositive-conditional} is the first example we have seen
of a mathematical statement that asserts something about the mathematical
constructs we have presented thus far.
When we make such an assertion, it is useful to demonstrate that our assertion
holds some clout.
To this end, we will introduce the notion of \definitionintro{truth tables}.

\begin{defn}
  A \definition{truth table}{truth table} shows all of the possible values
  of the propositional variables in at least one propositional statement,
  and the respective values of those propositional statements when those
  variables take on such values.

  Truth tables can be used to demonstrate the logical relationships between
  statements and demonstrate some useful property about those relationships.
\end{defn}

\begin{proof}[\thproofref{thm:contrapositive-conditional}]
  We will prove \thref{thm:contrapositive-conditional} using a truth table.
  \begin{table}[H]
    \centering
    \begin{tabular}{llcc}
      \toprule
      $p$ & $q$ & $p\implies q$ & $\neg q \implies \neg p$\\
      \midrule
      0 & 0 & 1 & 1 \\
      0 & 1 & 1 & 1 \\
      1 & 0 & 0 & 0 \\
      1 & 1 & 1 & 1 \\
      \bottomrule
    \end{tabular}
    \caption{A truth table for $p\implies q$ and $\neg q \implies \neg p$.}
    \label{tab:contrapositive}
  \end{table}
  In \tabref{tab:contrapositive}, we have shown that for all
  possible values of $p$ and $q$,
  the conditional statement $p \implies q$
  holds the same truth value as the contrapositive statement
  $\neg q \implies \neg p$.
\end{proof}

This property of contrapositives is especially useful when we are trying to prove a statement,
as its contrapositive is often easier to prove than the statement itself.
This subject is detailed further in \secref{sec:contrapositive}.

\index{inverse}
  The \textbf{inverse} of \(p \implies q\) is \(\neg p \implies \neg q\).
  It has the opposite truth value of the original statement.

  \index{biconditional statement}
  The \textbf{biconditional statement}, also called \emph{bi-implication}, for $p$ and $q$ is \[p \iff q.\] This is the statement
  ``\(p\) if and only if \(q\).''
  It is \ltrue{} when both \(p\) and \(q\) have the same truth value, and \lfalse{} otherwise.
\begin{table}[h]
  \centering
    \begin{tabular}{l}
      ``\(p\) is necessary and sufficient for \(q\)'' \\
      ``if \(p\), then \(q\), conversely'' \\
      ``p iff q''
    \end{tabular}
  \label{tab:biconditionals}
\end{table}
In mathematical theorems, definitions, and related elements later in this text, we will often see the biconditional operator written \emph{iff}.

\section{Logical Equivalence}
\index{logical equivalence}
\index{tautology}
  A \textbf{tautology} is a compound proposition that is always \ltrue{}, regardless of the truth values of the variables that occur in it.
\index{contradiction}
  A \textbf{contradiction} is a compound proposition that is always \lfalse{}.
\index{contingency}
  A \textbf{contingency} is a compound proposition that is neither a tautology nor a contradiction.
  It can be either \ltrue{} or \lfalse{}.
\begin{ex}
    \( p \iff q \) is logically equivalent to \( (p \implies q) \wedge (q \implies p)\).
    We state this by writing
    \[ p \iff q \equiv (p \implies q) \wedge (q \implies p).\]
    This is easiest to prove by truth table:
\begin{table}[H]
  \centering
    \begin{tabular}{lrcccc}
      \toprule
      $p$ & $q$ & $\neg q$ & $p \lor \neg q$ & $p \land q$ & $(p \lor \neg q) \implies (p \land q)$ \\ \midrule
      1 & 1 & 0      & 1             & 1         & 1 \\
      1 & 0 & 1      & 1             & 0         & 0 \\
      0 & 1 & 0      & 0             & 0         & 1 \\
      0 & 0 & 1      & 1             & 0         & 0 \\
      \bottomrule
    \end{tabular}
  \caption{The truth table for $(p \lor \neg q) \implies (p \land q)$}
\end{table}
\end{ex}
The compound propositions \(p\) and \(q\) are \textbf{logically equivalent} if \(p \iff q\) is a tautology.
The notation \(p \equiv q\) denotes that \(p\) and \(q\) are logically equivalent.
\begin{remark}
  The symbol \(\equiv\) is not a logical connection, and \(p\equiv q\) is not a compound proposition, but rather it is the statement that \(p \iff q\) is a tautology. While $\iff$ is an operator applied to propositional variables, $\equiv$ is an operator on entire propositional statements.
\end{remark}

\section{Precedence}
For more complex propositions, we need rules to tell which logical operators come first when we read them.

The \emph{not} operator ($\neg$) takes the highest precedence.
Then conjunction ($\land$), followed by disjunction ($\lor$).
Then conditional ($\implies$), and finally bi-conditional ($\iff$) operators.
\begin{table}[H]
  \centering
    \begin{tabular}{lll}
      \toprule
      precedence & symbol & meaning \\ \midrule
      1 & $\neg$      &not\\
      2 & $\land$     &and \\
      3 & $\lor$      & or \\
      4 & $\implies$  & conditional \\
      5 & $\iff$      &biconditional
      \\ \bottomrule
    \end{tabular}
  \caption{Logical operator precedence.}
  \label{tab:precedence}
\end{table}
\begin{ex}
    Is the statement $ p \land q \lor r $ equivalent to the statement
    \begin{enumerate}
        \item[a.] $(p \land q ) \lor r$,
        \item[b.] or rather, to $ p \land (q \lor r)$?
    \end{enumerate}
    \begin{sol}
        Let's look at \tabref{tab:precedence}.
        It states that $\land$ (``and'') operators come before $\lor$ (``or'') operators.
        Therefore,
        \[ p \land q \lor r \equiv (p \land q) \lor r,\]
        and (a) is the correct answer.
    \end{sol}
\end{ex}

\section{DeMorgan's Laws}
\textbf{DeMorgan's Laws} are an important concept in propositional logic and boolean algebra.

\begin{equation}
  \neg (p \land q) \equiv \neg p \lor \neg q
\end{equation}
\begin{equation}
  \neg(p \lor q) \equiv \neg p \land \neg q
\end{equation}
These can be proven using a \textbf{truth table}.
In order to construct a truth table, we must display all possible values of the propositional variables,
and the corresponding values of the propositional statement for each combination.
Intermediary steps may aid one's understanding, though they are not necessary in the final result.
\begin{table}[H]
  \centering
    \begin{tabular}{lrcc}
      \toprule
      $p$ & $q$ & $\neg(p \land q)$ & $\neg p \lor \neg q$ \\ \midrule
      0 & 0 & 1 & 1 \\
      0 & 1 & 1 & 1\\
      1 & 0 & 1 & 1\\
      1 & 1 & 0 & 0\\
      \bottomrule
    \end{tabular}
  \caption{A proof of DeMorgan's first law.}
\end{table}

\begin{table}[H]
  \centering
    \begin{tabular}{lrcc}
      \toprule
      $p$ & $q$ & $\neg(p \lor q)$ & $\neg p \land \neg q$ \\ \midrule
      0 & 0 & 1 & 1 \\
      0 & 1 & 0 & 0 \\
      1 & 0 & 0 & 0 \\
      1 & 1 & 0 & 0 \\
      \bottomrule
    \end{tabular}
  \caption{A proof of DeMorgan's second law.}
\end{table}

Extending DeMorgan's laws by the association laws for disjunction and conjunction, shown in \tabref{tab:logequiv}, they become:
\begin{equation}
 \neg\left(\bigwedge^n_{n=1} p_n\right)=\bigvee^n_{n=1} \neg p_n
\end{equation}
\begin{equation}
 \neg\left(\bigvee^n_{n=1} p_n\right)=\bigwedge^n_{n=1} \neg p_n
\end{equation}
Using DeMorgan's Laws, we can \emph{negate conjunctions and disjunctions}.
In computer science, we use DeMorgan's Laws to simplify boolean expressions.

\begin{ex}
  Negate the following statement:
  ``Miguel has a cellphone and he has a laptop.''
  \begin{sol}
    Let \(p\) be ``Miguel has a cell phone.''
    Let \(q\) be ``Miguel has a laptop.''
    The negation of $p \land q$ is
    \[ \neg p \lor \neg q, \]
    which means ``Miguel does not have a cell phone or he does not have a laptop.''
  \end{sol}
\end{ex}

\section{Useful Logical Equivalences}

In a proof or simplification of propositional logic statements, propositions like $p \implies q$
are difficult for us to work with.
We have very few laws or equivalences which work directly with them, so we often must convert them into an equivalent form using other operators.


We can convert $p\implies q$ to a proposition using only the $\neg$ and $\lor$ operators using the logical equivalence
\begin{equation}
  p \implies q \equiv \neg p \lor q,
\end{equation}
which we will prove using a truth table in \tabref{tab:conditionalproof}.
\begin{table}[H]
  \centering
    \begin{tabular}{lrcc}
      \toprule
      $p$ & $q$ & $p \implies q$ & $\neg p \lor q$ \\ \midrule
      1 & 1 & 1 & 1 \\
      1 & 0 & 0 & 0 \\
      0 & 1 & 1 & 1 \\
      0 & 0 & 1 & 1 \\
      \bottomrule
    \end{tabular}
  \caption{A proof of \(p \implies q \equiv \neg p \lor q\).}
  \label{tab:conditionalproof}
\end{table}

\begin{table}[H]
  \centering
    \begin{tabular}{lr}
      \toprule
      \textbf{Proposition} & \textbf{Name} \\ \midrule
      $p \land T \equiv p$ & \multirow{2}{*}{identity laws} \\
      $p \lor f \equiv p $ \\\midrule
      $p \lor T \equiv T $ & \multirow{2}{*}{domination laws} \\
      $p \land F \equiv F $ \\\midrule
      $p \lor p \equiv p $ & \multirow{2}{*}{idempotent laws} \\
      $p \land p \equiv p $ \\\midrule
      $\neg (\neg p) $ & double negation law \\ \midrule
      $p \lor q \equiv q \lor p $ & \multirow{2}{*}{commutative laws} \\
      $p \land q \equiv q \land p $ \\ \midrule
      $(p \lor q) \lor r \equiv p \lor (q \lor r) $& \multirow{2}{*}{associative laws} \\
      $(p \land q) \land r \equiv p \land (q \land r)$  \\\midrule
      $p \land (q \lor r) \equiv (p \land q) \lor (p \land r)$ & \multirow{2}{*}{distributive laws} \\
      $p \lor (q \land r) \equiv (p \lor q) \land (p \lor r)$ \\\midrule
      $\neg (p \lor q) \equiv \neg p \lor \neg q $ & \multirow{2}{*}{DeMorgan's laws} \\
      $\neg (p \lor q) \equiv \neg p \land \neg q $ \\\midrule
      $p \lor (p \land q) \equiv p $ & \multirow{2}{*}{absorbtion laws} \\
      $p \land (p \lor q) \equiv p $ \\\midrule
      $p \lor \neg p \equiv T $ & \multirow{2}{*}{negation laws} \\
      $p \land \neg p \equiv F $
    \end{tabular}
  \caption{Useful logical equivalence laws.}
  \label{tab:logequiv}
\end{table}

\section{Proving Logical Equivalences}

We can prove logical equivalences by using the rules in \tabref{tab:logequiv}, and showing
how one step leads to another until we reach something we know to be \ltrue{}.
Alternatively, we could start with a statement that we know to be \ltrue{} and work our way to the logical equivalence we are trying to prove.
Both of these proof methods are valid, though not especially rigorous, as they rely upon rules that we may not have established the truth value of with the mathematical rigor required to consider them \emph{formal proofs}.
\begin{ex}
  Show that \(\neg (p \implies q)\) and \( p \land \neg q\) are logically equivalent.
  \begin{sol}
    \[
      \begin{fitch}
        \fb \neg (p \implies q) \equiv \neg (p \implies q) \\
        \fa \neg(p \implies q) \equiv \neg(\neg p \lor q) & \(p \implies q \equiv \neg p \lor q\) \\
        \fa \neg(p \implies q) \equiv \neg(\neg p) \land \neg q & \text{DeMorgan's Law} \\
        \fa \neg(p \implies q) \equiv p \land \neg q & \text{Double negation}
      \end{fitch}
    \]
  \end{sol}
\end{ex}

Alternatively, they can be proven using truth tables, as described in \secref{def:contrapositive}.

% \section{Bit Operations}
%
% These logical operators can also be performed on \textbf{bits} (0 is \lfalse{}, 1 is \ltrue{}).
% These are called \textbf{bit operations}, and can even be performed on \textbf{bit strings}, sequences of zero or more bits.
% The \textbf{length} of a bit string is the number of bits in the string.
% \begin{ex}
%   Perform bitwise OR, AND, and XOR operations on the following bit strings.
%   \begin{align*}
%     01 \, 1011 \, 0100 & \\
%     11 \, 0001 \, 1101 &
%   \end{align*}
%   \begin{sol}
%     \begin{align*}
%       01 \, 1011 \, 0100 & \\
%       11 \, 0001 \, 1101 &
%       \\ \\
%       11 \, 1011 \, 1111 & \quad \text{bitwise OR} \\
%       01 \, 0001 \, 0100 & \quad \text{bitwise AND} \\
%       10 \, 1010 \, 1011 & \quad \text{bitwise XOR}
%     \end{align*}
%   \end{sol}
% \end{ex}

% \section{Examples}
%
% % bad example, copied from a textbook.
% % I need to write my own.
% \begin{ex}\cite[p.~14]{rosen}
%   Determine whether each of these conditional statements is \ltrue{} or \lfalse{}.
%   \begin{itemize}
%     \item[a) ] if \(1+1=2\), then \(2+2=5\)
%     \item[b) ] if \(1+1=3\), then \(2+2=4\)
%     \item[c) ] if \(1+1=3\), then \(2+2=5\)
%     \item[d) ] if monkeys can fly, then \(1+1=3\)
%   \end{itemize}
%   \begin{sol}
%     In each case, we simply determine the truth value of the hypothesis and the conclusion, then use the definition of the truth value of conditional statements to get our answer.
%     \begin{itemize}
%       \item[a) ] The hypothesis is \ltrue{}, and the conclusion is \lfalse{}, so the statement is \lfalse{}.
%       \item[b) ] Since the hypothesis is \lfalse{} and the conclusion is \ltrue{}, the statement is \ltrue{}.
%       \item[c) ] Since the hypothesis is \lfalse{}, regardless of the conclusion the conditional statement must be \ltrue{}.
%       \item[d) ] Since the hypothesis is \lfalse{}, the conditional is \ltrue{}.
%     \end{itemize}
%   \end{sol}
% \end{ex}

\section{Propositional Satisfiability}
% I'm pretty sure this section is ripped near-straight from a textbook too.
% I really need to write my own.

A compound proposition is \textbf{satisfiable} if there is an assignment of truth values to its variables that make it \ltrue{}.
When no such assignments exists, it is \textbf{unsatisfiable}.
A particular assignment of truth values that shows a compound proposition satisfiable is a \textbf{solution}.

To show a compound proposition is satisfiable, we need only demonstrate one possible solution.
However, to demonstrate it unsatisfiable, we would need to show every possible assignment of truth values and show why they make it \lfalse{}.
Thus, reasoning is often more useful than truth tables in showing a compound proposition is unsatisfiable.

% \begin{ex}
%   Determine whether each of the following is satisfiable.
%   \begin{itemize}
%     \item[a) ] \( (p \lor \neg q) \land (q \lor \neg r) \land (r \lor \neg p) \)
%     \item[b) ]  \((p \lor q \lor r) \land (\neg p \lor \neg q \lor \neg r) \)
%     \item[c) ]  \( (p \lor \neg q) \land (q \lor \neg r) \land (r \lor \neg p) \land (p \lor q \lor r) \land (\neg p \lor \neg q \lor \neg r)\)
%   \end{itemize}
%   \begin{sol}
%     (a) is \ltrue{} when \(p\), \(q\) and \(r\) have the same truth value.
%     (b) is \ltrue{} when at least one of \(p\), \(q\), or \(r\) is \lfalse{} and one is \ltrue{}.
%     So (c), a combination of (a) and (b), is unsatisfiable.
%   \end{sol}
% \end{exK

%%% Local Variables: 
%%% mode: latex
%%% TeX-master: "../notes"
%%% End: 
