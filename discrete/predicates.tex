\chapter{Predicates and Quantifiers}
\label{ch:predicates}
Predicates and quantifiers are an introduction to the idea that we can replace parts of our propositions with variables in order to separate our discussion of logic from the irrelevant details of the problem.

We saw in the previous chapter how the specifics of propositions seemed largely irrelavent; most examples were useless gibberish like ``my house is red'' with all of the focus on the theoretical relationships between these statements.

Predicate logic is a more powerful system of logic which allows us to state ``there exists'' and ``for all'' using logic.


\section{Predicate Logic}

Propositional logic is too simple for us to make many types of conclusions.
Instead, we use \textbf{predicate logic}, which allows us to make general
statements about objects and their properties.

A \textbf{propositional function}\index{propositonal function}, $P(x)$, is a
type of \textbf{predicate}\index{predicate} in
predicate logic. The important thing about propositional functions is that their truth value depends on the value of a variable, $x$.
A propositional function becomes a proposition when a value is assigned to $x$, and then it has a truth value and we can evaluate it.
\begin{ex}
  \[ P(n)=\text{``$n$ is prime''} \]
  \begin{remark}
    $P(n)$, a propositional function with a truth value, is different from the numerical function $p(n)$.
    When we talk about functions in the context of propositional logic, we must be careful not to confuse them with their possible numerical counterparts.
  \end{remark}
\end{ex}


\subsection{Domain of Discourse}\index{domain of discourse}
Just as values for a variable must be stated in order for a propositional function to have a truth value, a \textbf{domain of discourse}\index{domain of discourse} must be specified in addition to the universal quantification.
This is often referred to as just the \emph{domain} of the function.


For example, for porpostional functions talking about numbers, we often assume $D \to\mathbb{R}$.\footnote{We use $D$ as shorthand referring to the domain of discourse of a function.
$\mathbb{R}$ means ``all real numbers.'' That is, all rational and irrational numbers. It does not include, for example, complex numbers which include $i$, the ``imaginary unit.''}


\section{Quantification}\index{quantification}
%If we wish to state that a given propositional function is true for all possible values in a domain, we use the \emph{universal quantification} of that function.
%% Yeah, but we use this for so much more than that. Commenting this out for now.

The \textbf{universal quantification}\index{universal quantification} of $P(x)$
\begin{equation}
  \forall x P(x)
\end{equation}
is the statement
``$P(x)$ for all values of $x$ in the domain.''

To show that the universal quantification of $P(x)$ is false for a domain, simply find a single value of $x$ for which $P(x)$ is false.

\section{Existential Quantification}\index{existential quantification}

If we wish to state that an element exists in a domain, we use the \emph{existential quantification} of a propositional function.

The \textbf{existential quantificaiton}\index{existential quantification} of $P(x)$ is the proposition
  ``There exists an element $x$ in the domain such that $P(x)$.''
We use the notation \[\exists x P(x)\] for the existential quantification of $P(x)$.

\begin{note}
  In order to show that the existential quantification of $P(x)$ is false, we must
  show that $P(x)$ is false for every possible value of $x$ in the domain.
\end{note}


\subsection{Uniqueness Quantifier}
A specific case of existential quantification is defined by the
\textbf{uniqueness quantifier}\index{uniqueness quantifier}, $\exists!$ or $\exists_1$. The notation
\[ \exists! x P(x) \]
is the statement ``There exists a unique $x$ such that $P(x)$ is true.'' The
downside to the uniqueness quantifier is that the rules of inference for
existential quantification cannot be used on it. Since propositional logic can
be used to express uniqueness already, we should try to avoid use of uniqueness
quantification.

To demonstrate uniqueness using propositional logic, we make a statement such as the following:
\[ \exists x \Big( P(x) \land \forall y \big( P(y) \implies (x=y)\big)\Big) \]

\section{Logical Equivalence of Quantified Propositions}\index{logical equivalence}

In order for two statements involving predicates and quantifiers to be logically equivalent,
they must have the same truth value regardless of the values of their propositional variables
and the domain of discourse used.

DeMorgan's Laws are an important logical equivalence even when quantified propositions are discussed.
As stated in our definition of logial equivalence, they hold regardless of the values of their variables.

%The following is a quote from Rosen's \emph{Discrete Mathematics and its Applications} on the issue:
%
%\begin{quote}
%  Statements involving predicates and quantifiers are \textbf{logically
%  equivalent} if and only if they have the same truth value no matter what
%  predicates are substituted into the statements and which the domain of
%  discourse is used for the variables in these propositional functions. We use the
%  notation $S \equiv T$ to indicate that two statements $S$ and $T$ involving
%  predicates and quantifiers are logically equivalent.
%
%  \hfill\cite[p.~45]{rosen}
%\end{quote}

\subsection{DeMorgan's Laws for Quantifiers}\index{DeMorgan's laws for quantifiers}

DeMorgan's Laws for quantifiers allow us to radically simplify logical expressions involving quantifiers.

\begin{equation}
  \neg \exists x P(x) \equiv \forall x \neg P(x)
  \label{eq:dmq1}
\end{equation}
\begin{equation}
  \neg \forall x P(x) \equiv \exists x \neg P(x)
  \label{eq:dmq2}
\end{equation}

\begin{ex}
  For example, let's take \emph{Euler's conjecture},\footnote{Pronounced ``oiler.''} first proposed in 1769.\footnote{Eventually disproved in 1987. Solution at the end of the example.}

  Let us first define the propositional function $P(a,b,c,d)$.\footnote{ $ : : = $ is used to mean ``equals by definition,'' and is sometimes used in order to contrast with regular ``equals.''}
  \[P(a,b,c,d) : : =
    a^4 + b^4 + c^4 = d^4\]
Now, Euler proposed that there are no positive integers $a, b, c,$ and $d$ such that $P(a,b,c,d)$ is true. We state this by writing
  \[ E(a,b,c,d) : : =
    \forall a \in \mathbb{Z^+}
    \forall b \in \mathbb{Z^+}
    \forall c \in \mathbb{Z^+}
    \forall d \in \mathbb{Z^+}
    \big(\neg P(a,b,c,d) \big).\]
Let's break this apart.
The ``$a \in \mathbb{Z^+}$ is used to describe our \emph{domain of discourse}.
$\mathbb{Z^+}$ refers to the set of all positive integers.
In general use, we can simplify this statement by writing
\[ E(a,b,c,d) : : =
  \forall a,b,c,d, \in \mathbb{Z^+} \big( \neg P(x)\big)\]
but for our purposes, we want to work with the original proposition, because we wish to use DeMorgan's Laws on it.

Using DeMorgan's first law for quantifiers, equation \eqref{eq:dmq1}, we can change the last part of this proposition:
\[\forall d \in \mathbb{Z^+} \big(\neg P(a,b,c,d)\big)\equiv \neg \exists d \in \mathbb{Z^+} \big(P(a,b,c,d)\big)\]
Now, we continue up the chain, reversing each of the negated statements as if everything to the right of the negation sign were one single proposition.
Here's our new statement:
\begin{align*}
  E(a,b,c,d) : : &=
  \forall a \in \mathbb{Z^+}
  \forall b \in \mathbb{Z^+}
  \forall c \in \mathbb{Z^+}
  \neg\exists d \in \mathbb{Z^+}
  \big( P(a,b,c,d) \big)\\
  \intertext{Now, continuing DeMorgan's Laws,}
  E(a,b,c,d) : : &=
  \forall a \in \mathbb{Z^+}
  \forall c \in \mathbb{Z^+}
  \neg\exists c \in \mathbb{Z^+}
  \exists d \in \mathbb{Z^+}
  \big( P(a,b,c,d) \big)\\
  E(a,b,c,d) : : &=
  \forall a \in \mathbb{Z^+}
  \neg\exists c \in \mathbb{Z^+}
  \exists c \in \mathbb{Z^+}
  \exists d \in \mathbb{Z^+}
  \big( P(a,b,c,d) \big)\\
  \intertext{Arriving finally at}
  E(a,b,c,d) : : &=
  \neg\exists a \in \mathbb{Z^+}
  \exists c \in \mathbb{Z^+}
  \exists c \in \mathbb{Z^+}
  \exists d \in \mathbb{Z^+}
  \big( P(a,b,c,d) \big),\\
\end{align*}
which is logically equivalent to the original $E(a,b,c,d)$ we proposed.

This shows that if just one of the variables in $E(a,b,c,d)$ cannot be said to exist, then the entire proposition becomes false.

It turns out, in contrast to \emph{Euler's conjecture}, a solution to $P(a,b,c,d)$ can be found. With the values $a=95800$, $b=217519$, $c=414560$, and $d=422481$, $P(a,b,c,d)$ is true.
\end{ex}

\section{Order of Quantifiers}\index{quantifiers, order of}

Assuming a domain of discourse of all real numbers, the quantificaiton
\begin{equation}
  \exists y \forall x Q(x, y)
\end{equation}
denotes the propositon
``There is a real number $y$ such that for every real number $x$, $Q(x, y)$.''

By contrast, the quantificaiton
\begin{equation}
  \forall x \exists y Q(x, y)
\end{equation}
states that
``For every real number $x$ there is a real number $y$ such that $Q(x, y)$.''

