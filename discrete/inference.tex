\chapter{Rules of Inference}
  \epigraph{I don't want to believe. I want to know.}
{Carl Sagan}
Proofs are used to establish the truth of mathematical statements. In order to
make a proof, we must use the \textbf{rules of inference}\index{rules of
inference} to establish the truth of more complicated logical arguments. An
\textbf{argument} is a sequence of propositions that ends with a conclusion. A
\textbf{valid} argument is one in which the last proposition follows from those
propositions before it.

\section{Rules of Inference for Propositions}

\subsection{\emph{Modus ponens}}\label{modus_ponens}\index{\emph{modus ponens}}
\begin{equation*}
  \begin{fitch}
    \fb p       &\\
    \fa p \to q &\\
    \fa  q
  \end{fitch}
\end{equation*}
\emph{Modus ponens} is Latin for ``mode that affirms,'' and comes from the
tautology $(p \wedge (p \to q)) \to q$. It is the simplest valid
\textbf{argument}\index{argument}, a sequence of statements that ends with a conclusion.

\subsection{\emph{Modus tollens}}\index{\emph{modus tollens}}
\begin{equation*}
  \begin{fitch}
    \fb \neg q \\
    \fa p \to q \\
    \fa \neg p
  \end{fitch}
\end{equation*}

$(\neg q \wedge (p \to q)) \to \neg p$.

\subsection{Hypothetical syllogism}\index{hypothetical syllogism}
\begin{equation*}
  \begin{fitch}
    \fb p \to q \\
    \fa q \to r \\
    \fa p \to r
  \end{fitch}
\end{equation*}
Sometimes thought of as \emph{double modus ponens}, a hypothetical syllogism is
$((p \to q) \wedge (q \to r)) \to (p \to r)$.

\subsection{Disjunctive syllogism}\index{disjunctive syllogism}
\begin{equation*}
  \begin{fitch}
    \fb p \vee q \\
    \fa \neg p \\
    \fa q
  \end{fitch}
\end{equation*}
$(p \lor q ) \land \neg p \to q$.

\subsection{Addition}\index{addition}
\begin{equation*}
  \begin{fitch}
    \fb p \\
    \fa p \lor q
  \end{fitch}
\end{equation*}
$ p \to (p \lor q)$.

\subsection{Simplification}\index{simplification}
\begin{equation*}
  \begin{fitch}
    \fb p \land q \\
    \fa p
  \end{fitch}
\end{equation*}
$(p \land q) \to p$.

\subsection{Conjunction}\index{conjunction}
\begin{equation*}
  \begin{fitch}
    \fb p \\
    \fa q \\
    \fa p \land q
  \end{fitch}
\end{equation*}
$( (p) \land (q)) \to (p \land q)$.

\subsection{Resolution}\index{resolution}
\begin{equation*}
  \begin{fitch}
    \fb p \lor q \\
    \fa \neg p \lor r \\
    \fa q \lor r
  \end{fitch}
\end{equation*}
$ ( (p \lor q) \land (\neg p \lor r ) ) \to (q \lor r)$.



\section{Rules of Inference for Quantified Statements}

\subsection{Universal Generalization}
  \textbf{Universal generalization} states that given $P(c)$ for all elements $c$
  in the domain, $\forall x P(x)$ is true.
 \begin{equation}
  \begin{fitch}
    \fb P(c) \text{ for some arbitrary $c$} \\
    \fa \forall x P(x) & \text{Universal generalization}
  \end{fitch}
  \label{eq:univ_gen}
\end{equation}


\subsection{Universal Instantiation}\label{univ_inst}
\textbf{Universal instantiation}  states that given $\forall x P(x)$, $P(c)$ is
true for a particular element $c$ in the domain.
\begin{equation}
  \begin{fitch}
    \fb \forall x (P(x) \to Q(x)) \\
    \fa P(a) & \text{Universal instantiation}
  \end{fitch}
  \label{eq:univ_inst}
\end{equation}

\subsection{Existential Generalization}
  \textbf{Existential generalization}\index{existential generalization} concludes
  that, given a particular element $c$ for which $P(c)$ is known to be true, $\exists x P(x)$.


\subsection{Existential Instantiation}
  \textbf{Existential instatiation}\index{existential instatiation} states that if
  $\exists x P(x)$ is true, $P(c)$ for some element $c$.

\subsection{Universal \emph{Modus Ponens}}

\textbf{Universal \emph{modus ponens}} combines universal instantiation
(Section \ref{univ_inst}) and \emph{modus ponens} (Section \ref{modus_ponens}) to
tell us that if $\forall x (P(x) \to Q(x) )$ is true, and if $P(a)$ is true for a
particular element $a$ in the domain of the universal quantifier, then $Q(a)$ must
also be true.
\begin{equation}
  \begin{fitch}
    \fb \forall x (P(x) \to Q(x)) \\
    \fa P(a), \text{ where $a$ is a particular elment in the domain} \\
    \fa Q(a)
  \end{fitch}
  \label{eq:univ_mod_pon}
\end{equation}

\subsection{Universal \emph{Modus Tollens}}

\textbf{Universal \emph{modus tollens}} states that
\begin{equation}
  \begin{fitch}
    \fb \forall x (P(x) \to Q(x)) \\
    \fa \neg Q(a), \text{ where $a$ is a particular elment in the domain} \\
    \fa \neg P(a)
  \end{fitch}
  \label{eq:univ_mod_tol}
\end{equation}

