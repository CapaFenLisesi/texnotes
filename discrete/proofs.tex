\chapter{Proofs}\index{proofs}

Proofs are arguments that establish the truth of mathematical statements.

Most mathematical proofs for humans are \textbf{indirect proofs}\index{indirect
proofs}, where more
than one rule of inference may be used in each step, steps may be skipped, and
the axioms and rules of inference being used are not explicitly
state.\cite[p.~81]{rosen}

However, we may wish to teach a computer how to create proofs. In order to do
so, we must be extremely precise in our language. Likewise, in order to present
mathematically precise proofs, we must know how to write \emph{formal
proofs}\index{formal proofs}.

\begin{defn}
  A \textbf{fact} or \textbf{result} is a statement that can be shown to be
  true, but is not as important as a theorem.
\end{defn}
\begin{defn}
  A \textbf{lemma} is a less important theorem that is helpful in the proof of
  other theorems.
\end{defn}
\begin{defn}
  A \textbf{theorem} is a statement that can be shown to be true, associated
  with a degree of importance.
\end{defn}
\begin{defn}
  A \textbf{corollary} is a theorem that can be established directly from a
  theorem that has been provided.
\end{defn}
\begin{defn}
  A \textbf{conjecture} is a statement that is proposed to be true, but has not
  been proven to be true.
\end{defn}

%\section{Direct Proof}\index{direct proofs}
%
%A \textbf{direct proof} starts with the premises and follows directly to the
%conclusion.
%
%\begin{ex}
%  Give a direct proof of the theorem
%  \begin{quote}
%    ``If $n+1$ is an odd integer, then $n+3$ is an odd integer.''
%  \end{quote}
%  %\[ n+1 = 2k + 1 \]
%%   \begin{sol}
%%     \begin{proof}
%%       \begin{align*}
%%         n+1 &= 2k+1 & 2k+1 \text{ is odd.} \\
%%         n+3 &= 2k+1+3 & \text{Add }2\text{ to each side.} \\
%%         n+3 &= 2k+4 &\text{Simplify.}
%%       \end{align*}
%%     \end{proof}
%%   \end{sol}
%\end{ex}
%
%\section{Indirect Proof}\index{indirect proofs}
%
%An \textbf{indirect proof} is any proof that does not start with the premises
%and end with the conclusion. A \textbf{proof by contraposition} is a specific
%instance of indirect proof that starts with the contrapositive of the initial
%position and demonstrates its truth. Since the contrapositive of a statement is
%logically equivalent to the statement itself, this can be used to almost
%directly demonstrate the proof of the original statement.
%
%\begin{ex}
%  A perfect number\footnote{
%  	A \emph{perfect number} is one which is the sum of all its divisors except
%    itself. For example, 6 is perfect because $1+2+3=6$.
%  }is not a prime\footnote{A \emph{prime number} is a number whose only divisors
%are $1$ and itself.}.
%
%  \begin{proof}
%    First, we assume the number $p$ is prime $\to$ $p$ is not perfect.
%
%    Because the only divisors of a prime are $1$ and itself, the sum of the
%    divisors less than $p$ is $1$.
%
%    The sum of all divisors less than $1$, excluding $1$, is zero.
%
%    $p$ cannot be perfect.
%  \end{proof}
%\end{ex}
%\begin{ex}
%	``If $3n+2$ is odd, then $n$ is odd.''
%	\begin{proof}
%	If $3n+2$ is odd $\to$ $n$ is odd.
%	If $n$ is even $\to$ $3n+2$ is even.
%
%	Assume $n$ is even and $\to$ $n=2k$
%
%	$3n+2=3 \cdot 2k +2 = \underbrace{2(3k+1)}_{k'}=2k'$
%
%	$3n+2$ is even.
%	\end{proof}
%\end{ex}
%
%\section{Proof by Contradiction}\index{proof by contradiction}
%
%To prove $P$ is true, assume the conclusion $P$ is false. Derive a contradiction, usually of the form $R \wedge \neg R$ which establishes $\neg P \to 0$. The contrapositive of this assertion is $1 \to P$ from which it follows that $P$ must be true.
%
%\begin{equation}
%  \neg P \to \underbrace{(R \wedge \neg R)}_{\text{False.}}
%\end{equation}
%
%\begin{ex}
%	  There is no largest odd number.
%	\begin{proof}
%      $P$: There is no largest odd number. \\
%      $\neg P$: There is a largest odd number $x$. \\
%      $2x+1$ is odd, and $2x+1 > x \to \neg R$. \\
%      $\neg P \to (R \wedge \neg R)$ \\
%      $P$ is true.
%	\end{proof}
%\end{ex}
%\begin{ex}
%	  If $3n+2$ is odd, then $n$ is odd.
%    \[ P_1 \to P_2 = \neg P_1 \vee P_2 \]
%	  \[ \neg P = \neg(\neg P_1 \vee P_2) = \neg \neg P_1 \wedge \neg P_2 = P_1
%    \wedge \neg P_2 \]
%	\begin{proof}
%	    $P$: $3n+2$ is odd $\to$ $n$ is odd. \\
%      $P_1$: $3n+2$ is odd. \\
%      $P_2$ $n$ is odd. \\
%      $\neg P$: {$3n+2$ is odd}, and $n$ is not odd. \\
%      $n$ is even, $n=2k$ \\
%      $3n+2=3 \cdot 2k+2=2(3k+1)$ is {even}. \\
%      $\neg P$ is false. \\
%      $P$ is true.
%	\end{proof}
%\end{ex}
%\begin{ex}
%Suppose you wish to give a proof by contradiction of this result for
%\begin{center}
%  If $x$ and $y$ are odd, then $3x+2y$ is odd.
%\end{center}
%What do you begin by assuming?
%
%You assume that $P$ and $\neg Q$.
%\end{ex}
%
%
%\section{Proof by Cases}\index{proof by cases}
%
%To prove:
%  \[(P_1 \vee P_2 \vee \dots \vee P_n) \to Q \]
%Prove:
%  \[(P_1 \to Q) \wedge (P_2 \to Q) \wedge \dots \wedge (P_n \to Q) \]
%
%\begin{ex}
%	Prove that for $n \in N$, $n^3+3$ is even.
%	\begin{proof}
%	$n$ is even, $n=2k$. \\
%	$n^3+n=(2k)^3+2k=8k^3+2k=2(4k^3+k)$
%	$n$ is odd. $n=2k+1$.\\
%	$n^3+n=(2k+1)^3+(2k+1)=(2k+1)((2k+1)^2+1)$ \\
%	$=(2k+1)(4k^2+4k+2)$ \\
%	$=(2k+1)(2)(2k^2+2k+1)$ \\
%	$=2\underbrace{(\dots)(\dots)}_{k'}$
%	$n^3+n$ is even.
%  \end{proof}
%\end{ex}
%
%\section{Biconditional Statements}\index{biconditional statements}
%
%To prove a theorem that is a biconditional statement (a statement of the form $p \iff q$), we show that $p \to q$ and $q \to p$.
%
%\begin{ex}
%  ``$n$ is an odd integer iff $n^2$ is odd.''
%  \begin{proof}\footnote{Direct proof.}
%    $P$: $n$ is odd. \\
%    $Q$: $n^2$ is odd. \\
%    $P \to Q$: $n$ is odd, $n^2$ is odd. \\
%    Assume $n$ is odd, $n = 2k+1$ \\
%    $n^2$=$(2k+1)^2=4k^2+4k+1$ \\
%    $=2(2k^2+2k)+1$ \\
%    $n^2$ is odd.
%  \end{proof}
%  \begin{proof}\footnote{Indirect method of proof.}
%    $Q \to P$ \\
%    Assume $n$ is even, $n=2k$ \\
%    $n^2=(2k)^2=4k^2$ \\
%    $ 2(2k^2)$ \\
%    $n^2$ is even.
%    $P \iff Q$
%  \end{proof}
%\end{ex}
%\begin{homework}[Section 5.1]
%  pg. 329 3, 4, 7, 21
%
%
%  Chapter 1.6, 1.7, and 5.1 even due next Tuesday, March 20ish.
%  Quiz Thursday, March 13.
%\end{homework}
%
%
%
%
%\section{Mathematical Induction}\index{mathematical induction}
%
%\begin{defn}[Well-ordered]
%  A set $S$ is \emph{well-ordered} if every subset has a least element.
%  Integers and real numbers are not well-ordered sets. They do not contain a least element.
%\end{defn}
%
%Let $P(x)$ be a predicate over a well-ordered set S. The problem is to prove
%
%\[ \forall x P(x) \]
%
%We will prove this using the first principle of mathematical induction.
%
%\begin{remark}
%  From \emph{modus ponens},
%  \begin{align*}
%    &p
%    &p \to q
%    &
%  \end{align*}
%\end{remark}
%
%We might also derive triple \emph{modus ponens}, quadruple \emph{modus ponens}, etc. Thus, we have no trouble proving assertions about arbitrarily large integers.
%
%\begin{proof}
%\begin{align*}
%  &P(0) \\
%  &P(n) \to P(n+1) \\
%  &\forall x P(x) \\
%\end{align*}
%\end{proof}
%
%In this example, $P(O)$ forms \emph{the basis step}. The next part is called \emph{the induction}.
%
%We first prove the predicate is true for the smallest element of $S$, then apply \emph{modus ponens} an infinite number of times.
%
%To prove the basis step, merely verify that it is true for the sole case provided. To prove the induction, we normally use a \emph{direct proof}.
%
%\begin{proof}
%  Prove a classic:
%
%  \begin{tabular}{ll}
%    $\sum ^n_{i=0}i\frac{n(n+1)}{2} $ & Identify $P(n)$. \\
%    $\sum ^0_{i=0}i\frac{0(0+1)}{2}$ & Prove this to prove the basis step. \\
%    $\sum ^{n+1}_{i=0}i\frac{(n+1)((n+1 )+1)}{2} $ & Prove this using a direct proof.
%  \end{tabular}
%  \begin{align*}
%    P(n+1)=& 0+1+2+3+\dots+n+(n+1) \\
%    P(n+1)=&\frac{(n+1)\left( n+1 \right)+1}{2} \\
%    =&\frac{n(n+1)}{2}+(n+1) \\
%    =&\frac{(n+1)(n+2)}{n}
%  \end{align*}
%
%\end{proof}
%\begin{remark}
%  This is not a circular proof, because we are proving only that the conclusion follows from the hypotheses.
%\end{remark}
%
%\begin{ex}
%  Prove the sum of the first $n$ odd positive integers is $n^2 \forall (n>=1)$.
%
%  \begin{align*}
%    1+1=&1^2 \\
%    1+3=&2^2 \\
%    1+5=& 3^2 \\
%  \end{align*}
%
%  To prove this, first prove that $1=1^2$ then prove $P(n+1)$.
%
%  \begin{tabular}{ll}
%    $P(n+1)=1+3+5+\cdots+(2n-1)+(2n+1)$ & \\
%    $P(n)=1+3+5+\cdots+(2n-1)$ & $\to n^2+2n+1=(n+1)^2$\\
%  \end{tabular}
%
%\end{ex}
%
%\begin{ex}
%  Prove
%  \[ 2^n < n! \text{ for all positive integers } n >= 4 \]
%
%  \begin{tabular}{ll}
%    $p(n) : \forall n (n > 4 \to 2^n < n!)$ & \\
%    $p(4) : 2^4 < 4!$ & $16<24$ \\
%    $p(n+1) : 2^{n+1} < (n+1)!$ & Prove this as follows: \\
%    &$p(n+1):2^{n+1}=2\cdot2^n < 2\cdot n!$ \\
%    &Now we try to prove $2n! < (n+1)!$.\footnote{ noting that $(n+1)!=(n+1)n!$.} \\
%    & $2n! < (n+1)! = (n+1)n! \to$\\
%    &$2<n+1 \to$\\
%    &$1<n $ ?\\
%  \end{tabular}
%\end{ex}
%
%\begin{ex}
%  Prove that
%    \[ 2^{2n}-1 \text{ for } n>1 \text{ is divisible by } 3 \]
%  \begin{tabular}{lll}
%    $P(1):2^{2 \cdot 1} -1=3$& Basis.&Prove this is true. \\
%    $2^{2(n+1)}-1$ is divisible by $3$ & Induction. & Prove this. \\
%    && $2^{2n}-1=3 \cdot m \quad (m \in N)$ \\
%    && $2^{2(n+1)}=2^{2n+2}-1$ \\
%    && $ 2^{2(n+1)}=2^{2n}\cdot 4 -1 $ \\
%    && $2^{2(n+1)}=2^{2n}+3\cdot2^{2n}-1 $ \\
%    && $(2^{2(n+1)}=2^{2n}-1)+3 \cdot 2^{2n} $ \\
%    && Now we prove that both $2^{2n}-1$ and $3 \cdot 2^{2n}$ are divisible by 3. \\
%    && $(2^{2(n+1)}=3m+3\cdot2^{2n}$\\
%    && $ 3(m+2^{2n}$ \\
%    $2^{2(n+1)}-1$ & Induction conclusion. & \\
%  \end{tabular}
%\end{ex}
%
%
%
