\chapter{Proofs}\index{proofs}
We use proofs to establish the truth of mathematical statements.
There are a number of types of these statements, of varying importance.

Before we prove our statement, it is called a \textbf{conjecture}\index{conjecture}.

A \textbf{fact} or \textbf{result} is the simplest type of statement we might prove.
Sometimes proofs are required, but more often they are simply demonstrated, implied, or obvious.
\index{fact}\index{result}

\textbf{Theorems} are where proofs start to become important.
They are the foundations of complex mathematical logic.
 \index{theorem}
 Some theorems are less important than others, and we call these lemmata.\index{lemmata}
 A \textbf{lemma}\index{lemma} is a less important theorem, often used in proving our main theorem.
 After our theorem is proven, we can often draw \textbf{corollaries}\index{corollary}, theorems that follow easily from our proven theorem.

Some proofs are \textbf{formal proofs}, which use the rules of inference to establish truth.
These proofs are the ``simplest'' in that they involve the simplest forms of logic, though they can become incredibly complex as we attempt to write formal proofs for more and more complex theorems.

Thus, we most often will encounter \textbf{informal proofs}.
These proofs establish truth using language and logic that makes sense to humans.
They are often formed by building on

\section{Direct Proof}\index{direct proofs}

A \textbf{direct proof} starts with known facts, and the statement we are trying to prove
follows directly from them.

In using a direct proof for proving an implication \[P \implies Q\]
we start by assuming $P$ is true, then using the \emph{rules of inference}, reach the conclusion
$Q$ through direct logical steps from $P$.

% I have become fairly convinced this is meant to be proven by induction.
% \begin{ex}
%   Prove that if $n$ is greater than $4$ then $2^n$ is greater than $n^2$.
%   % A direct proof is easiest.
%   % Write a proof for this. It comes from one of my DM tests from freshman year, and I missed the points.
% \end{ex}
\begin{ex}
  Give a direct proof of the theorem
  \begin{quote}
    ``If $n+1$ is an odd integer, then $n+3$ is an odd integer.''
  \end{quote}
  \begin{proof}
    If $n+1$ is an odd integer, then we can write it in the form
    \begin{align*}
      n+1 &= 2k_1+1 \\
      \intertext{Now, we subtract $1$ from both sides.}
      n &= 2k_1 \\ \intertext{Which shows that $n$ is even. If we add $2$ to both sides,}
      n+2 &= 2k_1 +2 \\
      \intertext{An even number plus $2$ is still even, so we could write $2k_1+2$ as $2k_2$.}
      n+2 &= 2k_2 \\
      \intertext{Now we add $1$ to each side to reach our conclusion.}
      n+3 &= 2k_2 + 1
    \end{align*}
    Any number in the form $2k+1$ is odd.
    We have proven that if $n+1$ is an odd integer, then $n+3$ is also an odd integer.
  \end{proof}
\end{ex}

\section{Indirect Proof}\index{indirect proofs}

An \textbf{indirect proof} is, basically, any proof that is not a \emph{direct proof}.
There are a variety of ways of accomplishing a proof that are not direct proofs,
and many of them are dramatically simpler or easier to understand than the direct
proof for a given theorem would be.

\subsection{Proof by Contrapositive}\label{sec:contrapositive}
The first method of indirect proof we will discuss is the \textbf{proof by contrapositive}.
It is often more useful to work with the contrapositive something than the original statement.

Remembering our definition of the contrapositive from Section \ref{sec:propintro}, let's cover some examples.
\begin{defn}
  The \textbf{contrapositive} of the statement \(P \to Q \) is the statement \(\neg Q \to \neg P\).
\end{defn}

\begin{ex}
  Find the contrapositive of the following:
  \begin{quote}
    ``If it is snowing then it is cold.''
  \end{quote}
  \begin{sol}
    Assign propositional variables to both of the statements.
    \begin{quote}
      Let $s$ be ``it is snowing.'' Let $c$ be the statement ``it is cold.''
      The statement we are trying to prove is
      \[ P(x) : : = s\to c.\]
      Its contrapositive, therefore, is
      \[ C(x) : : = \neg c \to \neg s \]
      which is the statement ``if it is not cold then it is not snowing.''
      This is logically equivalent to our original statement.
      \[ P(x) \equiv C(x) \]
    \end{quote}
  \end{sol}
\end{ex}
\begin{ex}
  Prove the following, by contrapositive:
  \begin{quote}
  ``If a number squared is odd, then the number itself is odd.''
  \end{quote}
  \begin{proof}
    We will use proof by contraposition.
    \begin{quote}
      Let $S(x)$ be ``$x$ squared is odd.''

      Let $O(x)$ be ``$x$ is odd.''
    \end{quote}
    In mathematical language, this is\footnote{$2k+1$ is the simplest mathematical expression for a generic odd integer.}
    \begin{align*}
      S(x) & : : = x^2 = 2k_2+1 \\
      O(x) & : : = x = 2k_1+1
    \end{align*}
    where $k$ is any integer.

    So we are trying to prove
    \[ S(x) \to O(x). \]
    The contrapositive of this statement is
    \[\neg O(x) \to \neg S(x).\]
    Now, remembering that all even integers can be expressed in the form $n=2k$, let's find the negations of each of our propositional functions.
    \begin{align*}
      \neg S(x) &: : = x^2 = 2k_2 \\
      \neg O(x) &: : = x = 2k_1
    \end{align*}
    So the statement we are trying to prove is
    \begin{align*}
      \neg O(x) &\implies \neg S(x) \\
      x = 2k_1 &\implies x^2=2k_2
    \end{align*}
    From the first proposition,
    \begin{align*}
      x &= 2k_1 \\
      \intertext{Square both sides of the equation.}
      x^2 &= 2k_1^2 \\
      x^2 &= (2k_1)(2k_1) \\
      x^2 &= 4k_1 \\
      \intertext{Because all even integers are divisible by $2$, $4k_1$ could also be written as $2k_2$.}
      x^2 &= 2k_2
    \end{align*}
    which is the conclusion we were trying to reach.
    From this, we can conclude that
    \[ \neg O(x) \to \neg S(x)\]
    and therefore, by contrapositive,
    \[ S(x)\to O(x).\qedhere\]
  \end{proof}
\end{ex}


%
%An \textbf{indirect proof} is any proof that does not start with the premises
%and end with the conclusion. A \textbf{proof by contraposition} is a specific
%instance of indirect proof that starts with the contrapositive of the initial
%position and demonstrates its truth. Since the contrapositive of a statement is
%logically equivalent to the statement itself, this can be used to almost
%directly demonstrate the proof of the original statement.
%
%\begin{ex}
%  A perfect number\footnote{
%  	A \emph{perfect number} is one which is the sum of all its divisors except
%    itself. For example, 6 is perfect because $1+2+3=6$.
%  }is not a prime\footnote{A \emph{prime number} is a number whose only divisors
%are $1$ and itself.}.
%
%  \begin{proof}
%    First, we assume the number $p$ is prime $\to$ $p$ is not perfect.
%
%    Because the only divisors of a prime are $1$ and itself, the sum of the
%    divisors less than $p$ is $1$.
%
%    The sum of all divisors less than $1$, excluding $1$, is zero.
%
%    $p$ cannot be perfect.
%  \end{proof}
%\end{ex}
%\begin{ex}
%	``If $3n+2$ is odd, then $n$ is odd.''
%	\begin{proof}
%	If $3n+2$ is odd $\to$ $n$ is odd.
%	If $n$ is even $\to$ $3n+2$ is even.
%
%	Assume $n$ is even and $\to$ $n=2k$
%
%	$3n+2=3 \cdot 2k +2 = \underbrace{2(3k+1)}_{k'}=2k'$
%
%	$3n+2$ is even.
%	\end{proof}
%\end{ex}
%
\section{Proof by Contradiction}\index{proof by contradiction}

To use proof by contradiction to prove a proposition $P$, we start by assuming $P$ is false.
From this, we derive a logical contradiction.
Because $\neg P$ leads to a contradiction, we conclude that $P$ must be true.
%
%To prove $P$ is true, assume the conclusion $P$ is false. Derive a contradiction, usually of the form $R \wedge \neg R$ which establishes $\neg P \to 0$. The contrapositive of this assertion is $1 \to P$ from which it follows that $P$ must be true.
%
%\begin{equation}
%  \neg P \to \underbrace{(R \wedge \neg R)}_{\text{False.}}
%\end{equation}
%
%\begin{ex}
%	  There is no largest odd number.
%	\begin{proof}
%      $P$: There is no largest odd number. \\
%      $\neg P$: There is a largest odd number $x$. \\
%      $2x+1$ is odd, and $2x+1 > x \to \neg R$. \\
%      $\neg P \to (R \wedge \neg R)$ \\
%      $P$ is true.
%	\end{proof}
%\end{ex}
%\begin{ex}
%	  If $3n+2$ is odd, then $n$ is odd.
%    \[ P_1 \to P_2 = \neg P_1 \vee P_2 \]
%	  \[ \neg P = \neg(\neg P_1 \vee P_2) = \neg \neg P_1 \wedge \neg P_2 = P_1
%    \wedge \neg P_2 \]
%	\begin{proof}
%	    $P$: $3n+2$ is odd $\to$ $n$ is odd. \\
%      $P_1$: $3n+2$ is odd. \\
%      $P_2$ $n$ is odd. \\
%      $\neg P$: {$3n+2$ is odd}, and $n$ is not odd. \\
%      $n$ is even, $n=2k$ \\
%      $3n+2=3 \cdot 2k+2=2(3k+1)$ is {even}. \\
%      $\neg P$ is false. \\
%      $P$ is true.
%	\end{proof}
%\end{ex}
%\begin{ex}
%Suppose you wish to give a proof by contradiction of this result for
%\begin{center}
%  If $x$ and $y$ are odd, then $3x+2y$ is odd.
%\end{center}
%What do you begin by assuming?
%
%You assume that $P$ and $\neg Q$.
%\end{ex}
%
%
%\section{Proof by Cases}\index{proof by cases}
%
%To prove:
%  \[(P_1 \vee P_2 \vee \dots \vee P_n) \to Q \]
%Prove:
%  \[(P_1 \to Q) \wedge (P_2 \to Q) \wedge \dots \wedge (P_n \to Q) \]
%
%\begin{ex}
%	Prove that for $n \in N$, $n^3+3$ is even.
%	\begin{proof}
%	$n$ is even, $n=2k$. \\
%	$n^3+n=(2k)^3+2k=8k^3+2k=2(4k^3+k)$
%	$n$ is odd. $n=2k+1$.\\
%	$n^3+n=(2k+1)^3+(2k+1)=(2k+1)((2k+1)^2+1)$ \\
%	$=(2k+1)(4k^2+4k+2)$ \\
%	$=(2k+1)(2)(2k^2+2k+1)$ \\
%	$=2\underbrace{(\dots)(\dots)}_{k'}$
%	$n^3+n$ is even.
%  \end{proof}
%\end{ex}
%
%\section{Biconditional Statements}\index{biconditional statements}
%
%To prove a theorem that is a biconditional statement (a statement of the form $p \iff q$), we show that $p \to q$ and $q \to p$.
%
%\begin{ex}
%  ``$n$ is an odd integer iff $n^2$ is odd.''
%  \begin{proof}\footnote{Direct proof.}
%    $P$: $n$ is odd. \\
%    $Q$: $n^2$ is odd. \\
%    $P \to Q$: $n$ is odd, $n^2$ is odd. \\
%    Assume $n$ is odd, $n = 2k+1$ \\
%    $n^2$=$(2k+1)^2=4k^2+4k+1$ \\
%    $=2(2k^2+2k)+1$ \\
%    $n^2$ is odd.
%  \end{proof}
%  \begin{proof}\footnote{Indirect method of proof.}
%    $Q \to P$ \\
%    Assume $n$ is even, $n=2k$ \\
%    $n^2=(2k)^2=4k^2$ \\
%    $ 2(2k^2)$ \\
%    $n^2$ is even.
%    $P \iff Q$
%  \end{proof}
%\end{ex}
%\begin{homework}[Section 5.1]
%  pg. 329 3, 4, 7, 21
%
%
%  Chapter 1.6, 1.7, and 5.1 even due next Tuesday, March 20ish.
%  Quiz Thursday, March 13.
%\end{homework}
%
%
%
%
%\section{Mathematical Induction}\index{mathematical induction}
%
%\begin{defn}[Well-ordered]
%  A set $S$ is \emph{well-ordered} if every subset has a least element.
%  Integers and real numbers are not well-ordered sets. They do not contain a least element.
%\end{defn}
%
%Let $P(x)$ be a predicate over a well-ordered set S. The problem is to prove
%
%\[ \forall x P(x) \]
%
%We will prove this using the first principle of mathematical induction.
%
%\begin{remark}
%  From \emph{modus ponens},
%  \begin{align*}
%    &p
%    &p \to q
%    &
%  \end{align*}
%\end{remark}
%
%We might also derive triple \emph{modus ponens}, quadruple \emph{modus ponens}, etc. Thus, we have no trouble proving assertions about arbitrarily large integers.
%
%\begin{proof}
%\begin{align*}
%  &P(0) \\
%  &P(n) \to P(n+1) \\
%  &\forall x P(x) \\
%\end{align*}
%\end{proof}
%
%In this example, $P(O)$ forms \emph{the basis step}. The next part is called \emph{the induction}.
%
%We first prove the predicate is true for the smallest element of $S$, then apply \emph{modus ponens} an infinite number of times.
%
%To prove the basis step, merely verify that it is true for the sole case provided. To prove the induction, we normally use a \emph{direct proof}.
%
%\begin{proof}
%  Prove a classic:
%
%  \begin{tabular}{ll}
%    $\sum ^n_{i=0}i\frac{n(n+1)}{2} $ & Identify $P(n)$. \\
%    $\sum ^0_{i=0}i\frac{0(0+1)}{2}$ & Prove this to prove the basis step. \\
%    $\sum ^{n+1}_{i=0}i\frac{(n+1)((n+1 )+1)}{2} $ & Prove this using a direct proof.
%  \end{tabular}
%  \begin{align*}
%    P(n+1)=& 0+1+2+3+\dots+n+(n+1) \\
%    P(n+1)=&\frac{(n+1)\left( n+1 \right)+1}{2} \\
%    =&\frac{n(n+1)}{2}+(n+1) \\
%    =&\frac{(n+1)(n+2)}{n}
%  \end{align*}
%
%\end{proof}
%\begin{remark}
%  This is not a circular proof, because we are proving only that the conclusion follows from the hypotheses.
%\end{remark}
%
%\begin{ex}
%  Prove the sum of the first $n$ odd positive integers is $n^2 \forall (n>=1)$.
%
%  \begin{align*}
%    1+1=&1^2 \\
%    1+3=&2^2 \\
%    1+5=& 3^2 \\
%  \end{align*}
%
%  To prove this, first prove that $1=1^2$ then prove $P(n+1)$.
%
%  \begin{tabular}{ll}
%    $P(n+1)=1+3+5+\cdots+(2n-1)+(2n+1)$ & \\
%    $P(n)=1+3+5+\cdots+(2n-1)$ & $\to n^2+2n+1=(n+1)^2$\\
%  \end{tabular}
%
%\end{ex}
%
%\begin{ex}
%  Prove
%  \[ 2^n < n! \text{ for all positive integers } n >= 4 \]
%
%  \begin{tabular}{ll}
%    $p(n) : \forall n (n > 4 \to 2^n < n!)$ & \\
%    $p(4) : 2^4 < 4!$ & $16<24$ \\
%    $p(n+1) : 2^{n+1} < (n+1)!$ & Prove this as follows: \\
%    &$p(n+1):2^{n+1}=2\cdot2^n < 2\cdot n!$ \\
%    &Now we try to prove $2n! < (n+1)!$.\footnote{ noting that $(n+1)!=(n+1)n!$.} \\
%    & $2n! < (n+1)! = (n+1)n! \to$\\
%    &$2<n+1 \to$\\
%    &$1<n $ ?\\
%  \end{tabular}
%\end{ex}
%
%\begin{ex}
%  Prove that
%    \[ 2^{2n}-1 \text{ for } n>1 \text{ is divisible by } 3 \]
%  \begin{tabular}{lll}
%    $P(1):2^{2 \cdot 1} -1=3$& Basis.&Prove this is true. \\
%    $2^{2(n+1)}-1$ is divisible by $3$ & Induction. & Prove this. \\
%    && $2^{2n}-1=3 \cdot m \quad (m \in N)$ \\
%    && $2^{2(n+1)}=2^{2n+2}-1$ \\
%    && $ 2^{2(n+1)}=2^{2n}\cdot 4 -1 $ \\
%    && $2^{2(n+1)}=2^{2n}+3\cdot2^{2n}-1 $ \\
%    && $(2^{2(n+1)}=2^{2n}-1)+3 \cdot 2^{2n} $ \\
%    && Now we prove that both $2^{2n}-1$ and $3 \cdot 2^{2n}$ are divisible by 3. \\
%    && $(2^{2(n+1)}=3m+3\cdot2^{2n}$\\
%    && $ 3(m+2^{2n}$ \\
%    $2^{2(n+1)}-1$ & Induction conclusion. & \\
%  \end{tabular}
%\end{ex}


