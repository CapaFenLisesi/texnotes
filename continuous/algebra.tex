\chapter{Algebra}

\section{Laws}
If $a$, $b$, and $c$ are any numbers,\footnote{%
I am sure these all hold for real numbers, and presumably for complex as well,
though other number systems may have different laws.
I have not explored these possibilities.
}then the following laws hold:\footnote{%
Most of this is from the opening chapter of~\cite{spivak}, but bits and pieces
are collected from elsewhere and cited as such.
}
\subsection{Associative law for additon}
\index{addition!associative law}
The associative law extends our ability to discuss the operation $(+)$ on any 
two elements to three elements, without changing the order of these elements:
\begin{equation}
    a + ( b + c ) = (a + b) + c.
\end{equation}
It follows from this (though the proof is somewhat complicated, see~\cite[pg.~4]{spivak}),
that we may write sums without regard for parentheses.
This means that we may write, for instance
\begin{equation*}
    a_1 + a_2 + a_3 + a_4 + \cdots + a_n,
\end{equation*}
without any ambiguity as to what order the operation must be performed.

\subsection{Existence of an additive identity}
\index{addition!identity element}
The identity element for addition is 0.
This means that the sum of any element and 0 is always the original element.
We write this:
\begin{equation}
    a + 0 = 0 + a = a.
\end{equation}

\subsection{Existence of additive inverses}
\index{addition!inverse}
\begin{equation}
    a + (-a) = (-a) + a = 0.
\end{equation}
In this case, we mean that every element in the set $\mathbb{R}$ of real numbers
has an inverse with respect to the operation $(+)$.\cite[pg.~14]{pinter}

\subsection{Commutative law for addition}
\index{addition!commutative law}
\label{sec:comm:add}
This states that the value of $a + b$ or $b + a$ is independent of the order
in which $a$ and $b$ are taken.\cite[pg.~14]{pinter}
\begin{equation}
    a + b = b + a.
\end{equation}

\subsection{Associative law for multiplication}
\index{multiplication!associative law}
\begin{equation}
    a \cdot (b \cdot c) = (a \cdot b) \cdot c.
\end{equation}

\subsection{Existence of a multiplicative identity}
\index{multiplication!identity element}
\label{sec:mult:id}
Multiplication of real numbers has an identity element, $1$,
such that multiplying any number by this element gives us the original number:
\begin{equation}
    a \cdot 1 = 1 \cdot a = a, \qquad \text{for } 1 \neq 0.
    \label{eq:mult:id}
\end{equation}
The notation here is a little strange.
We know that $1$ is the identity element for multiplication, but it also
refers to the number $1$, so why do we state that $1 \neq 0$?
Of course one is not equal to zero!

The reason for this is that we are talking about the \emph{element} 1, this
being the identity element for multiplication, and not simply the \emph{number}
1. We may just as well have written:
\begin{equation*}
    a \cdot e = e \cdot a = a, \qquad \text{for } e \neq 0,
\end{equation*}
but writing 1 instead of $e$ as in \eref{eq:mult:id} here makes sense,
since 1 is, in fact, both the number and the element in question.

\subsection{Existence of multiplicative inverses}
For every element $a$ in $\mathbb{R}$, there is an element $a^{-1}$ in 
$\mathbb{R}$ such that $a \cdot a^{-1}$ gives us the identity element
from \secref{sec:mult:id}.
\begin{equation}
    a \cdot a^{-1} = a^{-1} \cdot a = 1, \qquad \text{for } a \neq 0.
\end{equation}

\subsection{Commutative law for multiplication}
As with the commutative law for addition (\secref{sec:comm:add}),
this states that the value of $a * b$ or $b * a$ is independent of the order
in which $a$ and $b$ are taken.\cite[pg.~14]{pinter}
\begin{equation}
    a \cdot b = b \cdot a.
\end{equation}

\subsection{Distributative law}
\index{distributive law}
The distributive law is a relationship between multiplication and addition.
It allows us to manipulate the order of application when we are combining these
two operations.
\begin{equation}
    a \cdot (b + c) = a \cdot b + a \cdot c.
\end{equation}

\section{Inequality}
\index{inequalities}
When we say $a$ is 
\emph{less than}\index{inequalities!less than}
$b$, we write $a < b$,
and take it to mean the same thing as saying $b$ is 
\emph{greater than}\index{inequalities!greater than}
$a$ ($b > a$).\cite[pg.~9]{spivak}
Thus the numbers $a$ satisfying $a > 0$ are called \emph{positive}, while
those numbers $a$ satisfying $a < 0$ are called \emph{negative}.

\section{More laws}
\begin{theorem}[Trichotomy law]
\index{trichotomy law}
For every number $a$, one and only one of the following holds:
\begin{enumerate}
    \item $a = 0$,
    \item $a$ is in the collection $P$,
    \item $-a$ is in the collection $P$.
\end{enumerate}
\cite[pg.~9]{spivak}
\end{theorem}

\begin{theorem}[Closure under addition]
\index{addition!closure}
If $a$ and $b$ are in $P$, then $a + b$ is in $P$.
\cite[pg.~9]{spivak}
\end{theorem}

\begin{theorem}[Closure under multiplication]
\index{multiplication!closure}
If $a$ and $b$ are in $P$, then $a \cdot b$ is in $P$.
\cite[pg.~9]{spivak}
\end{theorem}

\begin{defn}
\index{absolute value}
    For any number $a$, we define the \emph{absolute value}\index{absolute value}
    $|a|$ of $a$ to be:\cite[pg.~11]{spivak}
    \begin{equation}
        |a| = \begin{dcases}
            a, &a \geq 0\\
            -a, & a \leq 0.
        \end{dcases}
    \end{equation}
\end{defn}
