\chapter{Ordinary Differential Equations}
\begin{defn}
  A \textbf{differential equation} is an equation involving derivatives.
\end{defn}
\begin{defn}
  A \textbf{direction field} tells us the slope of a function at any given place.
\end{defn}
\begin{ex}
    In physics, we often define acceleration to be a vector relative to another vector, velocity.
    Here, we will just consider them as scalars for the sake of argument.
    Accleration is a change in velocity, so
    \begin{equation}
        a = \leib{v}{t},
        \label{eq:acceleration}
    \end{equation}
    where $v$ represents velocity and $t$ represents time.
    Now we integrate both sides of \eref{eq:acceleration} with respect to $t$,
    \begin{equation}
        \int a \ud t = \int \leib{v}{t} \ud t.
        \label{eq:intaccel}
    \end{equation}
    To simplify things, let us assume $\ud t$ in the denominator of the right-hand side of \eref{intaccel} and the freestanding $\ud t$ are simply variables and may be cancelled as such.
    We cannot actually do this (as ``$\ud$'' is an operator and not a variable, and it cannot be treated as one), but the results turn out to have interesting implications:
    \begin{equation}
        \int a \ud t = \int \ud v.
        \label{eq:naughtint}
    \end{equation}
    From here we simplify, finding that
    \begin{equation}
        at + c_1 = v + c_2.
        \label{eq:almostvelocity}
    \end{equation}
    The constants in \eref{eq:almostvelocity} are simply constants and may be combined into another constant, $C$.
    Also, the equation may be rearranged to put it in more familiar form, yielding
    \begin{equation}
        v = at + C,
        \label{eq:velocity}
    \end{equation}
    which we recognize as the classical mechanics equation for velocity.
    Integrating once more, we find
    \begin{align*}
        \int v \ud t &= \int (at + C) \ud t, \\
        \int v \ud t &= \int at \ud t + \int C \ud t, \\
        vt + c_3  &= a \frac{t^2}{2} + Ct + c_4. \\
        \intertext{Now we may simply combine the constants once more, defining $C_1$ to constitute the sum of $c_3$, and $c_4$.}
        vt &= a \frac{t^2}{2} + Ct + C_1
    \end{align*}
\end{ex}

\section{Integrating Factors}
We will observe differential equations of the form
\begin{equation}
  \leib{y}{t} = g(t) y + r(t)
\end{equation}
\begin{enumerate}
  \item $\leib{y}{t} = t^2 y + \cos{t}$
  \item $t y +3=\leib{y}{t}-2t$
\end{enumerate}
